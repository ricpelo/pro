\usepackage{etoolbox}

% Los bloques de código se muestran enmarcados:
\usepackage[xparse,skins,breakable]{tcolorbox}

% El texto monoespaciado se muestra de otro color:
\definecolor{ttcolor}{RGB}{38,139,210}
\let\Oldtexttt\texttt
\renewcommand\texttt[1]{\textcolor{ttcolor}{\Oldtexttt{#1}}}
\let\Oldmathtt\mathtt
\renewcommand\mathtt[1]{\textcolor{ttcolor}{\Oldmathtt{#1}}}

\ifdefined\Shaded
  \renewenvironment{Shaded}{\centering \begin{caja}}{\end{caja}}
\fi

% Adjust fontsize for code-blocks if necessary (by redefining the
% Highlighting-Environment of the pandoc-template)
\ifdefined\Highlighting
  \DefineVerbatimEnvironment{Highlighting}{Verbatim}{commandchars=\\\{\},fontsize=\footnotesize}
\fi

% Hace lo mismo con los entornos verbatim:
\makeatletter
\patchcmd{\@verbatim}
  {\verbatim@font}
  {\verbatim@font\footnotesize}
  {}{}
\makeatother

% Activa ligaduras en verbatim:
\makeatletter
\def\verbatim@nolig@list{}
\makeatother

%\makeatletter
%\let \clone@tabularcr \@tabularcr
%\def\@tabularcr{\clone@tabularcr\hline}
%\makeatother
%\arrayrulewidth=0.01mm

% Deja más espacio entre filas de tablas:
\renewcommand{\arraystretch}{1.5}

% Apaño para que Pandoc traduzca lo que hay entre \begin{...} y \end{...}:
\newcommand{\hideFromPandoc}[1]{#1}
\hideFromPandoc{
    \let\Begin\begin
    \let\End\end
}

% No muestra "Figura #: ":
\usepackage{caption}
\captionsetup[figure]{labelformat=empty}

% Las listas se numeran estilo 1.2.3.4:
\usepackage[shortlabels]{enumitem}
\setlist[enumerate]{label*=\arabic*.,leftmargin=*}
\setlist[enumerate,2]{label=\alph*),font=\itshape}

% Se puede hacer un salto de línea tras :: en texto monoespaciado
% (útil, por ejemplo, en yii\base\Component::__construct()):
\usepackage{xparse}
\ExplSyntaxOn
\NewDocumentCommand{\replace}{mmm}
 {
  \marian_replace:nnn {#1} {#2} {#3}
 }
\tl_new:N \l_marian_input_text_tl
\cs_new_protected:Npn \marian_replace:nnn #1 #2 #3
 {
  \tl_set:Nn \l_marian_input_text_tl { #1 }
  \tl_replace_all:Nnn \l_marian_input_text_tl { #2 } { #3 }
  \tl_use:N \l_marian_input_text_tl
 }
\ExplSyntaxOff
\usepackage{url}
\let\OldTexttt\texttt
\renewcommand{\texttt}[1]{\OldTexttt{\replace{#1}{::}{::\allowbreak}}}

% Preguntas y respuestas:
\usepackage[answerdelayed]{exercise}
\renewcommand{\ExerciseName}{Pregunta}
\renewcommand{\AnswerName}{Respuesta}
\renewcommand{\ExerciseHeader}{\textbf{\large%
\ExerciseName\ \ExerciseHeaderNB\ExerciseHeaderTitle%
\ExerciseHeaderOrigin\medskip}\par}
\renewcommand{\AnswerHeader}{\textbf{%
Respuesta a la \ExerciseName\ \ExerciseHeaderNB}\smallskip\par}

% Para que funcione la marca ✔:
\usepackage{newunicodechar}
\newfontface{\libertinus}{LibertinusSerif-Regular.otf} % a font that has ✔
\newunicodechar{✔}{%
  \begingroup
  \normalfont\libertinus ✔%
  \endgroup
}
\newunicodechar{␣}{·}
\newunicodechar{⟨}{%
  \begingroup
  \normalfont\libertinus ⟨%
  \endgroup
}
\newunicodechar{⟩}{%
  \begingroup
  \normalfont\libertinus ⟩%
  \endgroup
}

% Tipografías para matemáticas:
%\usepackage{unicode-math}
%\setmathsfont(Latin,Greek)[ItalicFont=LinBiolinum_RI.otf,
%    BoldFont=LinBiolinum_RB.otf]{LinBiolinum_R.otf}
\setmathsfont(Greek){XITSMath-Regular.otf}
\setmathsfont(Latin,Digits){Lato}
\setallmonofonts[Extension=.otf,UprightFont=*-Regular,BoldFont=*-Bold,AutoFakeSlant,BoldItalicFeatures={FakeSlant},Scale=MatchLowercase,Contextuals={Alternate}]{FiraMono}
%\setmathfont[range=\mathit]{Lato Italic}

\usepackage[yyyymmdd,hhmmss]{datetime}

% Resuelve un problema que aparece en las últimas versiones de TexLive que hace
% que salga este mensaje de error:
% Error producing PDF.
% ! Undefined control sequence.
% \@currentHref ->table.\theHtable
%
% l.2 \phantomsection\label{ejemplos}
%\newcommand{\theHtable}{\thetable}
