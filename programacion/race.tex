\begin{center}
\footnotesize
\begin{longtable}[c]{|>{\raggedright}m{4cm}|>{\centering}m{0.7cm}|>{\centering}m{0.7cm}|>{\centering}m{0.7cm}|>{\centering}m{0.7cm}|>{\centering}m{0.7cm}|>{\centering}m{0.7cm}|>{\centering}m{0.7cm}|>{\centering}m{0.7cm}|>{\centering}m{0.7cm}|}
\hline
\textbf{Unidades didácticas} & \ra1 & \ra2 & \ra3 & \ra4 & \ra5 & \ra6 & \ra7 & \ra8 & \ra9\tabularnewline
\hline
\hline
\endhead
1. Introducción &  &  &  &  &  &  &  &  &  \tabularnewline
\hline
2. Lenguajes de programación &  &  &  &  &  &  &  &  &  \tabularnewline
\hline
3. Expresiones &  &  &  &  &  &  &  &  &  \tabularnewline
\hline
4. Programación funcional (I) &  &  &  &  &  &  &  &  &  \tabularnewline
\hline
5. Programación funcional (II) &  &  &  &  &  &  &  &  &  \tabularnewline
\hline
6. Abstracciones funcionales &  &  &  &  &  &  &  &  &  \tabularnewline
\hline
7. Evaluación &  &  &  &  &  &  &  &  &  \tabularnewline
\hline
8. Programación imperativa (I) &  &  &  &  &  &  &  &  &  \tabularnewline
\hline
9. Programación imperativa (II) &  &  &  &  &  &  &  &  &  \tabularnewline
\hline
10. Programación estructurada &  &  &  &  &  &  &  &  &  \tabularnewline
\hline
11. Programación procedimental &  &  &  &  &  &  &  &  &  \tabularnewline
\hline
12. Calidad &  &  &  &  &  &  &  &  &  \tabularnewline
\hline
13. Colecciones e iteradores &  &  &  &  &  &  &  &  &  \tabularnewline
\hline
14. Secuencias &  &  &  &  &  &  &  &  &  \tabularnewline
\hline
15. Colecciones no secuenciales &  &  &  &  &  &  &  &  &  \tabularnewline
\hline
16. Entrada y salida por archivos &  &  &  &  &  &  &  &  &  \tabularnewline
\hline
17. Programación modular &  &  &  &  &  &  &  &  &  \tabularnewline
\hline
18. Abstracción de datos &  &  &  &  &  &  &  &  &  \tabularnewline
\hline
19. Programación orientada a objetos &  &  &  &  &  &  &  &  &  \tabularnewline
\hline
20. Relaciones entre clases &  &  &  &  &  &  &  &  &  \tabularnewline
\hline
21. Programación de interfaces gráficas de usuario &  &  &  &  &  &  &  &  &  \tabularnewline
\hline
22. Bases de datos orientadas a objetos &  &  &  &  &  &  &  &  &  \tabularnewline
\hline
23. Bases de datos relacionales &  &  &  &  &  &  &  &  & \ce{9a}\ \ce{9b}\ \ce{9c}\ \ce{9d}\ \ce{9e}\ \ce{9f}\ \ce{9g}\ \tabularnewline
\hline
\end{longtable}
\par\end{center}
