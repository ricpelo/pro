\begin{center}
\footnotesize
\begin{longtable}[c]{|>{\raggedright}m{4cm}|>{\centering}m{0.7cm}|>{\centering}m{0.7cm}|>{\centering}m{0.7cm}|>{\centering}m{0.7cm}|>{\centering}m{0.7cm}|>{\centering}m{0.7cm}|>{\centering}m{0.7cm}|>{\centering}m{0.7cm}|>{\centering}m{0.7cm}|}
\hline
\textbf{Unidades didácticas} & \ra1 & \ra2 & \ra3 & \ra4 & \ra5 & \ra6 & \ra7 & \ra8 & \ra9\tabularnewline
\hline
\hline
\endhead
1. Introducción & \ce{1a}\ & \ce{2a}\ &  &  &  &  &  &  &  \tabularnewline
\hline
2. Lenguajes de programación & \ce{1b}\ \ce{1c}\ & \ce{2a}\ &  &  &  &  &  &  &  \tabularnewline
\hline
3. Expresiones & \ce{1d}\ \ce{1e}\ \ce{1f}\ \ce{1g}\ & \ce{2b}\ \ce{2c}\ \ce{2d}\ \ce{2e}\ \ce{2f}\ \ce{2g}\ \ce{2h}\ \ce{2i}\ &  &  &  &  &  &  &  \tabularnewline
\hline
4. Tipos y operaciones & \ce{1d}\ \ce{1f}\ \ce{1g}\ \ce{1h}\ &  &  &  &  &  &  &  &  \tabularnewline
\hline
5. Programación funcional & \ce{1e}\ \ce{1f}\ \ce{1i}\ &  &  &  &  &  &  &  &  \tabularnewline
\hline
6. Abstracciones funcionales &  &  &  & \ce{4d}\ &  &  &  &  &  \tabularnewline
\hline
7. El modelo de entorno & \ce{1d}\ \ce{1e}\ \ce{1f}\ \ce{1g}\ &  &  &  &  &  &  &  &  \tabularnewline
\hline
8. Programación imperativa (I) & \ce{1e}\ &  &  &  &  &  &  &  &  \tabularnewline
\hline
9. Programación imperativa (II) &  &  &  &  & \ce{5a}\ \ce{5c}\ &  &  &  &  \tabularnewline
\hline
10. Programación estructurada &  &  & \ce{3a}\ \ce{3b}\ \ce{3c}\ \ce{3d}\ \ce{3e}\ \ce{3f}\ \ce{3g}\ \ce{3i}\ &  &  &  &  &  &  \tabularnewline
\hline
11. Programación procedimental &  &  & \ce{3a}\ \ce{3b}\ \ce{3c}\ \ce{3d}\ \ce{3e}\ \ce{3f}\ \ce{3g}\ \ce{3h}\ \ce{3i}\ &  &  &  &  &  &  \tabularnewline
\hline
12. Procesamiento de datos de alto nivel &  &  &  &  &  & \ce{6b}\ \ce{6c}\ \ce{6e}\ \ce{6f}\ \ce{6g}\ &  &  &  \tabularnewline
\hline
13. Secuencias &  &  &  &  & \ce{5b}\ & \ce{6a}\ \ce{6b}\ \ce{6c}\ \ce{6d}\ \ce{6h}\ &  &  &  \tabularnewline
\hline
14. Colecciones no secuenciales &  &  &  &  &  & \ce{6b}\ \ce{6c}\ \ce{6i}\ \ce{6j}\ &  &  &  \tabularnewline
\hline
15. Entrada y salida por archivos &  &  &  &  & \ce{5c}\ \ce{5d}\ \ce{5e}\ &  &  &  &  \tabularnewline
\hline
16. Programación modular &  &  &  & \ce{4i}\ &  &  &  &  &  \tabularnewline
\hline
17. Abstracción de datos &  &  &  & \ce{4b}\ \ce{4e}\ &  &  &  &  &  \tabularnewline
\hline
18. Programación orientada a objetos (I) &  &  & \ce{3h}\ & \ce{4a}\ \ce{4b}\ \ce{4c}\ \ce{4d}\ \ce{4e}\ \ce{4f}\ \ce{4g}\ \ce{4i}\ &  & \ce{6g}\ &  &  &  \tabularnewline
\hline
19. Programación orientada a objetos (II) &  &  & \ce{3h}\ & \ce{4a}\ \ce{4b}\ \ce{4c}\ \ce{4d}\ \ce{4e}\ \ce{4f}\ \ce{4g}\ \ce{4i}\ &  & \ce{6g}\ &  &  &  \tabularnewline
\hline
20. Relaciones entre clases &  &  &  & \ce{4a}\ \ce{4b}\ \ce{4c}\ \ce{4d}\ \ce{4e}\ \ce{4f}\ \ce{4g}\ \ce{4h}\ \ce{4i}\ &  &  & \ce{7a}\ \ce{7b}\ \ce{7c}\ \ce{7d}\ \ce{7e}\ \ce{7f}\ \ce{7g}\ \ce{7h}\ \ce{7i}\ \ce{7j}\ &  &  \tabularnewline
\hline
21. Interfaces gráficas de usuario &  &  &  &  & \ce{5f}\ \ce{5g}\ \ce{5h}\ &  &  &  &  \tabularnewline
\hline
22. Bases de datos orientadas a objetos &  &  &  &  &  &  &  & \ce{8a}\ \ce{8b}\ \ce{8c}\ \ce{8d}\ \ce{8e}\ \ce{8f}\ \ce{8g}\ \ce{8h}\ &  \tabularnewline
\hline
23. Bases de datos relacionales &  &  &  &  &  &  &  &  & \ce{9a}\ \ce{9b}\ \ce{9c}\ \ce{9d}\ \ce{9e}\ \ce{9f}\ \ce{9g}\ \tabularnewline
\hline
\end{longtable}
\par\end{center}
