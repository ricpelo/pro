\begin{longenum}
    \item \textbf{\textsc{Introducción}} \ev1\ (Est: 2019-09-16)
    \begin{longenum}
        \item Funcionamiento básico de un ordenador
        \begin{longenum}
            \item Elementos funcionales
            \item Ciclo de instrucción
            \item Representación de información
            \begin{longenum}
                \item Codificación interna
                \begin{longenum}
                    \item Sistema binario
                \end{longenum}
                \item Codificación externa
                \begin{longenum}
                    \item ASCII
                    \item Unicode
                \end{longenum}
            \end{longenum}
        \end{longenum}
        \item Resolución de problemas mediante programación
        \begin{longenum}
            \item Análisis del problema
            \item Especificación
            \item Diseño del algoritmo
            \item Codificación del algoritmo en forma de programa
        \end{longenum}
        \item Conceptos básicos
        \begin{longenum}
            \item Algoritmo
            \begin{longenum}
                \item Concepto
                \item Características
                \item Representación
                \begin{longenum}
                    \item Ordinograma
                    \item Pseudocódigo
                \end{longenum}
                \item Cualidades deseables
                \item Computabilidad
                \item Corrección
                \item Complejidad
            \end{longenum}
            \item Programa
            \item Lenguaje de programación
        \end{longenum}
        \item Evolución histórica
        \begin{longenum}
            \item Culturas de la programación
            \item Ingeniería del software
        \end{longenum}
        \item Paradigmas de programación
        \begin{longenum}
            \item Imperativo
            \begin{longenum}
                \item Estructurado
                \item Orientado a objetos
            \end{longenum}
            \item Declarativo
            \begin{longenum}
                \item Funcional
                \item Lógico
            \end{longenum}
        \end{longenum}
        \item Lenguajes de programación
        \begin{longenum}
            \item Definición
            \begin{longenum}
                \item Sintaxis
                \begin{longenum}
                    \item Notación EBNF
                \end{longenum}
                \item Semántica
            \end{longenum}
            \item Evolución histórica
            \item Clasificación
            \begin{longenum}
                \item Por nivel
                \item Por generación
                \item Por paradigma
            \end{longenum}
        \end{longenum}
        \item Traductores
        \begin{longenum}
            \item Compiladores
            \item Intérpretes
            \begin{longenum}
                \item Interactivos (\textit{REPL})
                \item Por lotes
            \end{longenum}
        \end{longenum}
        \item El lenguaje de programación PHP
        \begin{longenum}
            \item Historia
            \item Características principales
        \end{longenum}
        \item Entornos integrados de desarrollo
        \begin{longenum}
            \item Definición y tipos. Entornos comerciales y de Software libre.
            \item Instalación y descripción de entornos integrados de desarrollo.
            \item Creación de proyectos. Estructura y componentes.
        \end{longenum}
    \end{longenum}
    \item \textbf{\textsc{Programación funcional I}} \ev1\ (Est: 2019-09-23)
    \begin{longenum}
        \item Modelo de ejecución
        \begin{longenum}
            \item Evaluación de expresiones
            \item Modelo de sustitución / sistema de reescritura
        \end{longenum}
        \item Expresiones
        \begin{longenum}
            \item Valores, expresión canónica y forma normal
            \item Literales
            \item Operaciones, operadores y operandos
            \begin{longenum}
                \item Precedencia y asociatividad de operadores
            \end{longenum}
            \item Evaluación
            \begin{longenum}
                \item Orden de evaluación
                \begin{longenum}
                    \item Orden aplicativo
                    \item Orden normal
                \end{longenum}
            \end{longenum}
            \item Tipos de datos
            \begin{longenum}
                \item Concepto
                \begin{longenum}
                    \item Tipo de un valor
                    \item Tipo de una expresión
                \end{longenum}
                \item Tipos de datos básicos
                \begin{longenum}
                    \item Números
                    \begin{longenum}
                        \item Operadores aritméticos
                    \end{longenum}
                    \item Cadenas
                \end{longenum}
            \end{longenum}
            \item Algebraicas vs. algorítmicas
            \item Aritméticas
            \item Funciones predefinidas
            \item Constantes predefinidas
        \end{longenum}
        \item Álgebra de Boole
        \begin{longenum}
            \item El tipo de dato \textit{booleano}
            \item Operadores relacionales
            \item Operadores lógicos
            \item Axiomas
            \item Propiedades
            \item El operador ternario
        \end{longenum}
        \item Sentencias
        \begin{longenum}
            \item Variables
            \begin{longenum}
                \item Identificadores
                \item Ligadura (\textit{binding})
                \item Estado
                \item Asignación no destructiva
                \item Tipado estático vs. dinámico
            \end{longenum}
            \item Evaluación de expresiones con variables
        \end{longenum}
        \item Autodocumentación
        \begin{longenum}
            \item Comentarios
            \item Reglas de estilo
        \end{longenum}
    \end{longenum}
    \item \textbf{\textsc{Programación funcional II}} \ev1\ (Est: 2019-09-30)
    \begin{longenum}
        \item Abstracciones funcionales
        \begin{longenum}
            \item Definición de funciones anónimas
            \item Parámetros y argumentos
            \item Paso de argumentos por valor
            \item Ámbito de las variables
            \item La sentencia \texttt{return}
        \end{longenum}
        \item Composición de funciones
        \item Computabilidad
        \begin{longenum}
            \item Funciones recursivas
            \item Un lenguaje Turing-completo
        \end{longenum}
        \item Tipos de datos compuestos
        \begin{longenum}
            \item Las cadenas como datos compuestos
            \item Los \textit{arrays} como listas inmutables de elementos
        \end{longenum}
        \item Funciones de orden superior
        \begin{longenum}
            \item \texttt{array\_map()}
            \item \texttt{array\_filter()}
            \item \texttt{array\_reduce()}
            \item Funciones locales a funciones
            \item Funciones anónimas
        \end{longenum}
        \item \textit{Scripts}
    \end{longenum}
    \item \textbf{\textsc{Programación imperativa}} \ev1\ (Est: 2019-10-07)
    \begin{longenum}
        \item Modelo de ejecución
        \begin{longenum}
            \item Máquina de estados
            \item Secuencia de instrucciones
        \end{longenum}
        \item Cambios de estado explícitos
        \begin{longenum}
            \item Celdas
            \item Asignación destructiva (o asignación múltiple)
            \item Asignación por referencia
        \end{longenum}
        \item Efectos laterales
        \begin{longenum}
            \item Transparencia referencial
            \item Entrada y salida por consola
            \begin{longenum}
                \item La sentencia \texttt{echo}
                \item Las funciones \texttt{var\_dump()} y \texttt{print\_r()}
                \item Las funciones \texttt{fgets()} y \texttt{fscanf()}
            \end{longenum}
        \end{longenum}
        \item Saltos
        \begin{longenum}
            \item Incondicionales: la sentencia \texttt{goto}
            \item Condicionales: la sentencia \texttt{if (...) goto}
            \item Implementación de bucles mediante saltos condicionales
        \end{longenum}
        \item Los \textit{arrays} como estructura de datos mutable básica
        \begin{longenum}
            \item Creación, acceso y modificación
            \item Recorrido y búsqueda en un \textit{array}
            \item \textit{Arrays} multidimensionales
            \item Funciones de manejo de \textit{arrays}
        \end{longenum}
    \end{longenum}
    \item \textbf{\textsc{Programación estructurada}} \ev1\ (Est: 2019-10-14)
    \begin{longenum}
        \item Teorema de Böhm-Jacopini
        \item Estructuras básicas de control
        \begin{longenum}
            \item Secuencia
            \item Selección
            \item Iteración
        \end{longenum}
        \item Recursos abstractos
        \item Diseño descendente
        \item Refinamiento sucesivo
    \end{longenum}
    \item \textbf{\textsc{Tipos de datos}} \ev1\ (Est: 2019-10-21)
    \begin{longenum}
        \item Introducción
        \item Tipos básicos
        \begin{longenum}
            \item Lógicos (\texttt{bool})
            \begin{longenum}
                \item Operadores lógicos
            \end{longenum}
            \item Numéricos
            \begin{longenum}
                \item Enteros (\texttt{int})
                \item Números en coma flotante (\texttt{float})
                \item Operadores
                \begin{longenum}
                    \item Operadores aritméticos
                    \item Operadores de incremento/decremento
                \end{longenum}
            \end{longenum}
            \item Nulo (\texttt{null})
        \end{longenum}
        \item Tipos compuestos
        \begin{longenum}
            \item \textit{Arrays} asociativos
            \begin{longenum}
                \item Operadores para arrays
                \begin{longenum}
                    \item Acceso, modificación y agregación
                \end{longenum}
                \item Funciones de manejo de arrays
                \begin{longenum}
                    \item Ordenación de arrays
                    \item \texttt{print\_r()}
                    \item \texttt{'+'} vs. \texttt{array\_merge()}
                    \item \texttt{isset()} vs. \texttt{array\_key\_exists()}
                \end{longenum}
                \item \texttt{foreach}
                \item Conversión a \texttt{array}
                \item \textit{Ejemplo}: \texttt{\$argv} en CLI
            \end{longenum}
            \item Cadenas (\texttt{string})
            \begin{longenum}
                \item Operadores de cadenas
                \begin{longenum}
                    \item Concatenación
                    \item Acceso y modificación por caracteres
                    \item Operador de incremento \opcional\
                \end{longenum}
                \item Funciones de manejo de cadenas
                \item Expresiones regulares
                \item Extensión \textit{mbstring}
            \end{longenum}
            \item Callables
            \begin{longenum}
                \item \texttt{call\_user\_func()}
                \item \texttt{array\_map()} y \texttt{array\_reduce()}
            \end{longenum}
            \item Iterable
        \end{longenum}
        \item Manipulación de tipos
        \begin{longenum}
            \item Operadores de asignación compuesta
            \item Comprobaciones
            \begin{longenum}
                \item De tipos
                \begin{longenum}
                    \item \texttt{gettype()}
                    \item \texttt{is\_*()}
                \end{longenum}
                \item De valores
                \begin{longenum}
                    \item \texttt{is\_numeric()}
                    \item \texttt{ctype\_*()}
                \end{longenum}
            \end{longenum}
            \item Conversiones de tipos
            \begin{longenum}
                \item Conversión explícita (forzado o \textit{casting}) vs. automática
                \item Conversión a \texttt{bool}
                \item Conversión a \texttt{int}
                \item Conversión a \texttt{float}
                \item Conversión de \texttt{string} a número
                \item Conversión a \texttt{string}
                \item Funciones de obtención de valores
                \begin{longenum}
                    \item \texttt{intval()}
                    \item \texttt{floatval()}
                    \item \texttt{strval()}
                    \item \texttt{boolval()}
                \end{longenum}
                \item Funciones de formateado numérico
                \begin{longenum}
                    \item \texttt{number\_format()}
                    \item \texttt{money\_format()}
                    \begin{longenum}
                        \item \texttt{setlocale()}
                    \end{longenum}
                \end{longenum}
            \end{longenum}
            \item Comparaciones
            \begin{longenum}
                \item Operadores de comparación
                \item \texttt{==} vs. \texttt{===}
                \item Ternario (\texttt{?:})
                \item Fusión de \texttt{null} (\texttt{??})
                \item Reglas de comparación de tipos
            \end{longenum}
        \end{longenum}
    \end{longenum}
    \item \textbf{\textsc{Metodología de la programación}} \ev1\ (Est: 2019-10-28)
    \begin{longenum}
        \item Ciclo de vida
        \item Especificación e implementación
        \item Verificación y validación de programas
        \begin{longenum}
            \item Demostración por inducción
        \end{longenum}
        \item Programación funcional
        \begin{longenum}
            \item Especificaciones formales
            \begin{longenum}
                \item Como cálculo
            \end{longenum}
            \item Derivación de programas
            \begin{longenum}
                \item Diseño recursivo
                \begin{longenum}
                    \item Procesos recursivos e iterativos
                    \item Recursividad final
                    \item Técnicas de inmersión
                \end{longenum}
            \end{longenum}
        \end{longenum}
        \item Programación imperativa
        \begin{longenum}
            \item Especificaciones formales
            \begin{longenum}
                \item Como modificación de estados
            \end{longenum}
            \item Derivación de programas
            \begin{longenum}
                \item Diseño iterativo
                \begin{longenum}
                    \item Invariante de un bucle
                    \item Transformación de recursividad final a iterativo
                \end{longenum}
            \end{longenum}
        \end{longenum}
        \item Depuración
        \begin{longenum}
            \item Depuración de programas
            \item El depurador como herramienta de control de errores
        \end{longenum}
    \end{longenum}
    \item \textbf{\textsc{Programación modular I}} \ev1\ (Est: 2019-11-04)
    \begin{longenum}
        \item Introducción
        \begin{longenum}
            \item Descomposición de problemas
        \end{longenum}
        \item Funciones definidas por el usuario
        \begin{longenum}
            \item Definición de funciones con \texttt{function}
            \item Paso de argumentos
            \begin{longenum}
                \item Por valor
                \item Por referencia
            \end{longenum}
            \item Ámbito de variables
            \begin{longenum}
                \item Variables globales
                \begin{longenum}
                    \item Ámbito simple al archivo
                    \item Efectos laterales
                \end{longenum}
                \item Variables locales
            \end{longenum}
            \item Declaraciones de tipos
            \begin{longenum}
                \item Declaraciones de tipo de argumento
                \item Declaraciones de tipo de devolución
                \item Tipos \textit{nullable} (\texttt{?}) y \texttt{void}
                \item Tipificación estricta
            \end{longenum}
        \end{longenum}
        \item Partes de un módulo
        \begin{longenum}
            \item Interfaz
            \item Implementación
            \item Documentación interna
        \end{longenum}
        \item Inclusión de \textit{scripts}
        \begin{longenum}
            \item \texttt{require} y \texttt{require\_once}
            \item \texttt{include} e \texttt{include\_once}
        \end{longenum}
        \item Espacios de nombres \opcional\
        \item Criterios de descomposición modular
        \begin{longenum}
            \item Abstracción
            \item Ocultación de información
            \item Independencia funcional
            \begin{longenum}
                \item Cohesión
                \item Acoplamiento
            \end{longenum}
            \item Reusabilidad
        \end{longenum}
        \item Diagramas de estructura
    \end{longenum}
    \item \textbf{\textsc{Complejidad algorítmica}} \ev1\ (Est: 2019-11-11)
    \begin{longenum}
        \item Introducción
        \item Principio de invarianza
        \item La notación asintótica \textit{O(f(n))}
        \item Órdenes de complejidad
        \item Operaciones entre órdenes de complejidad
        \begin{longenum}
            \item Regla de la suma
            \item Regla del producto
        \end{longenum}
        \item Reglas prácticas para el cálculo de la eficiencia
        \item Resolución de recurrencias
        \begin{longenum}
            \item Reducción de problemas mediante sustracción
            \item Reducción de problemas mediante división
        \end{longenum}
    \end{longenum}
    \item \textbf{\textsc{Entrada y salida de información}} \ev1\ (Est: 2019-11-18)
    \begin{longenum}
        \item Flujos
        \begin{longenum}
            \item Tipos de flujos
            \begin{longenum}
                \item Flujos de bytes
                \item Flujos de caracteres
            \end{longenum}
            \item Utilización de flujos
        \end{longenum}
        \item La consola
        \begin{longenum}
            \item Entrada desde teclado
            \item Salida a pantalla
        \end{longenum}
        \item Archivos de datos
        \begin{longenum}
            \item Registros
            \item Apertura y cierre de archivos
            \item Modos de acceso
            \item Lectura y escritura de información en archivos
        \end{longenum}
        \item Sistemas de archivos
        \begin{longenum}
            \item Manipulación de los sistemas de archivos
            \item Creación y eliminación de archivos y directorios
        \end{longenum}
    \end{longenum}
    \item \textbf{\textsc{Pruebas y depuración}} \ev1\ (Est: 2019-11-25)
    \begin{longenum}
        \item Depuración
        \begin{longenum}
            \item \texttt{var\_dump()}
            \item Depuración con PsySH
        \end{longenum}
        \item Pruebas
        \begin{longenum}
            \item Tipos de pruebas
            \begin{longenum}
                \item Unitarias
                \item Funcionales
                \item De aceptación
            \end{longenum}
            \item PHPUnit
            \item Cobertura de código \opcional\
        \end{longenum}
    \end{longenum}
    \item \textbf{\textsc{Introducción al lenguaje Java}} \ev2\ (Est: 2020-01-15)
    \begin{longenum}
        \item Introducción
        \item Compilación vs. interpretación
        \begin{longenum}
            \item Máquinas reales vs. virtuales
            \item Código objeto, \textit{bytecode} y archivos \texttt{.class}
            \item La plataforma Java
            \begin{longenum}
                \item La máquina virtual de Java (JVM)
                \item La API de Java
            \end{longenum}
            \item El entorno de ejecución de Java (JRE)
            \begin{longenum}
                \item El intérprete \texttt{java}
            \end{longenum}
            \item Las herramientas de desarrollo de Java (JDK)
            \begin{longenum}
                \item El compilador \texttt{javac}
            \end{longenum}
        \end{longenum}
        \item Características de Java
        \item Tipado estático vs. dinámico
        \item El programa \textit{¡Hola, mundo!} en Java
        \begin{longenum}
            \item El método \texttt{main()}
            \item La clase \texttt{System}
        \end{longenum}
        \item Tipos de datos primitivos
    \end{longenum}
    \item \textbf{\textsc{Programación orientada a objetos}} \ev2\ (Est: 2020-01-22)
    \begin{longenum}
        \item Introducción
        \begin{longenum}
            \item Perspectiva histórica
            \item Lenguajes orientados a objetos
        \end{longenum}
        \item Conceptos básicos
        \begin{longenum}
            \item Clase
            \item Objeto
            \begin{longenum}
                \item La antisimetría dato-objeto
            \end{longenum}
            \item Identidad
            \item Estado
            \item Propiedad
            \item Paso de mensajes
            \item Método
            \item Encapsulación
            \item Herencia
            \item Polimorfismo
        \end{longenum}
        \item Uso básico de objetos
        \begin{longenum}
            \item Instanciación
            \begin{longenum}
                \item \texttt{new}
                \item \texttt{instanceof}
            \end{longenum}
            \item Propiedades
            \begin{longenum}
                \item Acceso y modificación
            \end{longenum}
            \item Referencias
            \item Clonación de objetos
            \item Comparación de objetos
            \item Destrucción de objetos
            \begin{longenum}
                \item Recolección de basura
            \end{longenum}
            \item Métodos
            \item Constantes
            \item \textit{Ejemplo}: Las cadenas (clase \texttt{String})
        \end{longenum}
        \item Lenguaje UML
        \begin{longenum}
            \item Diagramas de clases
            \item Diagramas de objetos
            \item Diagramas de secuencia
        \end{longenum}
    \end{longenum}
    \item \textbf{\textsc{Diseño de clases}} \ev2\ (Est: 2020-01-29)
    \begin{longenum}
        \item Encapsulación
        \item Propiedades
        \begin{longenum}
            \item Visibilidad
            \begin{longenum}
                \item Pública
                \item Privada
            \end{longenum}
        \end{longenum}
        \item Métodos
        \begin{longenum}
            \item Visibilidad
            \begin{longenum}
                \item Pública
                \item Privada
            \end{longenum}
            \item Referencia \texttt{this}
            \item Sobrecarga
            \item Constructores y destructores
            \item Accesores y mutadores
        \end{longenum}
        \item Constantes
        \item Miembros estáticos
        \begin{longenum}
            \item Constantes
            \item Métodos estáticos
            \item Propiedades estáticas
        \end{longenum}
    \end{longenum}
    \item \textbf{\textsc{Composición, herencia y poliformismo}} \ev2\ (Est: 2020-02-05)
    \begin{longenum}
        \item Composición de clases
        \item Herencia
        \begin{longenum}
            \item Concepto de herencia
            \item Tipos
            \item Superclases y subclases
            \item Visibilidad protegida
            \item Utilización de clases heredadas
            \item Constructores y herencia
            \item \texttt{super}
            \item Clases y métodos abstractos
            \item Clases y métodos finales
        \end{longenum}
        \item Polimorfismo
        \begin{longenum}
            \item El principio de sustitución de Liskov
            \item Sobreescritura de métodos
            \item Sobreescritura de constructores
        \end{longenum}
        \item Herencia vs. composición
    \end{longenum}
    \item \textbf{\textsc{Abstracción de datos}} \ev2\ (Est: 2020-02-12)
    \begin{longenum}
        \item Tipos abstractos de datos
        \begin{longenum}
            \item Concepto, terminología y ejemplos
            \item Programación con tipos abstractos de datos
            \begin{longenum}
                \item Modularidad
                \item Refinamientos sucesivos
                \item Programación a gran escala
                \item Programación genérica
            \end{longenum}
        \end{longenum}
        \item Especificación
        \begin{longenum}
            \item Especificaciones algebraicas
            \begin{longenum}
                \item Signatura de un TAD
                \begin{longenum}
                    \item Géneros
                    \item Operaciones
                    \begin{longenum}
                        \item Constructoras
                        \item Selectoras
                    \end{longenum}
                \end{longenum}
                \item Términos
                \item Ecuaciones
            \end{longenum}
            \item Construcción de especificaciones
            \item Verificación con especificaciones algebraicas
        \end{longenum}
        \item Implementación
        \begin{longenum}
            \item Pilas
            \item Colas
            \item Listas
        \end{longenum}
    \end{longenum}
    \item \textbf{\textsc{Programación modular II}} \ev2\ (Est: 2020-02-19)
    \begin{longenum}
        \item Las clases como módulos
        \begin{longenum}
            \item Interfaz de una clase
            \item Métodos \textit{getter} y \textit{setter}
        \end{longenum}
        \item Interfaces
        \begin{longenum}
            \item Definición de interfaces
            \item Implementación de interfaces
            \item Las interfaces como tipos
            \item Métodos predeterminados
        \end{longenum}
        \item Paquetes y módulos
    \end{longenum}
    \item \textbf{\textsc{Estructuras de datos lineales}} \ev2\ (Est: 2020-02-26)
    \begin{longenum}
        \item Acceso directo
        \begin{longenum}
            \item \textit{Arrays} en java
        \end{longenum}
        \item Acceso secuencial
        \begin{longenum}
            \item Listas
            \begin{longenum}
                \item Enlazadas
                \item Doblemente enlazadas
            \end{longenum}
            \item Pilas
            \item Colas
        \end{longenum}
    \end{longenum}
    \item \textbf{\textsc{Ordenación y búsqueda}} \ev2\ (Est: 2020-03-04)
    \begin{longenum}
        \item Algoritmos de búsqueda
        \begin{longenum}
            \item Búsqueda secuencial
            \item Búsqueda dicotómica
        \end{longenum}
        \item Algoritmos de ordenación
        \begin{longenum}
            \item Inserción directa
            \item Selección directa
            \item Burbuja
            \item \textit{Quicksort}
            \item \textit{Mergesort}
        \end{longenum}
        \item Tablas \textit{Hash}
    \end{longenum}
    \item \textbf{\textsc{Estructuras de datos no lineales}} \ev2\ (Est: 2020-03-11)
    \begin{longenum}
        \item Árboles
        \begin{longenum}
            \item Binarios
            \begin{longenum}
                \item Recorridos
                \begin{longenum}
                    \item Preorden
                    \item Inorden
                    \item Postorden
                \end{longenum}
            \end{longenum}
            \item De búsqueda
            \item Montículos
            \begin{longenum}
                \item Algoritmo de ordenación
            \end{longenum}
            \item Generales
            \begin{longenum}
                \item Recorrido en profundidad
                \item Recorrido en anchura
            \end{longenum}
        \end{longenum}
        \item Grafos
        \begin{longenum}
            \item Algoritmo de Dijkstra
            \item Algoritmo de Floyd
        \end{longenum}
    \end{longenum}
    \item \textbf{\textsc{Control de excepciones}} \ev2\ (Est: 2020-03-18)
    \begin{longenum}
        \item Errores y excepciones
        \item El requisito «\textit{captura o especifica}»
        \begin{longenum}
            \item Tipos de excepciones
        \end{longenum}
        \item Captura y manejo de excepciones
        \begin{longenum}
            \item Bloque \texttt{try}
            \item Bloques \texttt{catch}
            \item Bloque \texttt{finally}
        \end{longenum}
        \item Excepciones y signaturas
        \item Lanzamiento de excepciones
        \begin{longenum}
            \item Excepciones encadenadas
            \item Creación de clases de excepción
        \end{longenum}
        \item Excepciones no chequeadas
        \item Ventajas de las excepciones
    \end{longenum}
    \item \textbf{\textsc{Principios y patrones de diseño}} \ev2\ (Est: 2020-03-25)
    \begin{longenum}
        \item Principios de diseño
        \begin{longenum}
            \item Encapsulación y ocultación de información
            \item Diseño orientado a interfaces
            \item Principios \textit{SOLID}
            \begin{longenum}
                \item SRP: Principio de responsabilidad única
                \item OCP: Principio de abierto/cerrado
                \item LSP: Principio de sustitución de Liskov
                \item ISP: Principio de segregación de la interfaz
                \item DIP: Principio de inversión de dependencias
            \end{longenum}
            \item Principio del Menor Conocimiento (o Ley de Demeter)
        \end{longenum}
        \item Patrones de diseño
        \begin{longenum}
            \item De creación
            \item Estructurales
            \item De comportamiento
        \end{longenum}
    \end{longenum}
    \item \textbf{\textsc{Java Collection Framework}} \ev3\ (Est: 2020-04-13)
    \item \textbf{\textsc{Calidad}} \ev1\ (Est: 2020-04-20)
    \begin{longenum}
        \item Pruebas automáticas
        \begin{longenum}
            \item Tipos de pruebas
            \begin{longenum}
                \item Unitarias
                \item Funcionales
                \item De aceptación
            \end{longenum}
            \item JUnit
            \item Cobertura de código \opcional\
        \end{longenum}
        \item Documentación
        \item Mantenimiento y calidad del código
        \begin{longenum}
            \item CodeSniffer
            \item CS\_Fixer
        \end{longenum}
    \end{longenum}
    \item \textbf{\textsc{Gestión de bases de datos relacionales}} \ev3\ (Est: 2020-04-27)
    \begin{longenum}
        \item Componentes de acceso a datos
        \begin{longenum}
            \item Clase \texttt{PDO}
            \begin{longenum}
                \item \texttt{\_\_construct(string \$dsn [, string \$username [, string \$password [, array \$options ]]])}
                \item \texttt{PDOStatement query(string \$statement)}
                \item \texttt{int exec(string \$statement)}
                \item \texttt{PDOStatement prepare(string \$statement [, array \$driver\_options = array() ])}
            \end{longenum}
            \item Clase \texttt{PDOStatement}
            \begin{longenum}
                \item \texttt{mixed fetch([ int \$fetch\_style ])}
                \item \texttt{mixed fetchAll([ int \$fetch\_style ])}
                \item \texttt{mixed fetchColumn([ int \$column\_number = 0 ])}
                \item \texttt{bool execute ([ array \$input\_parameters ])}
                \item \texttt{int rowCount(void)}
            \end{longenum}
            \item Correspondencias de tipos entre SQL y PHP
        \end{longenum}
        \item Establecimiento de conexiones
        \item Recuperación de información
        \begin{longenum}
            \item Ejecución de consultas
            \item Selección de registros
            \item Uso de parámetros
        \end{longenum}
        \item Manipulación de la información
        \begin{longenum}
            \item Altas, bajas y modificaciones
        \end{longenum}
    \end{longenum}
\end{longenum}
