\begin{longenum}
    \item \textbf{\textsc{Introducción}} \ev1\ \ra1\ (est: 2019\==09\==17)
    \begin{longenum}
        \item Conceptos básicos
        \begin{longenum}
            \item Informática
            \begin{longenum}
                \item Procesamiento automático
            \end{longenum}
            \item Ordenador
            \begin{longenum}
                \item Definición
                \item Funcionamiento básico
                \begin{longenum}
                    \item Elementos funcionales
                    \item Ciclo de instrucción
                    \item Representación de información
                    \begin{longenum}
                        \item Codificación interna
                        \begin{longenum}
                            \item Sistema binario
                        \end{longenum}
                        \item Codificación externa
                        \begin{longenum}
                            \item ASCII
                            \item Unicode
                        \end{longenum}
                    \end{longenum}
                \end{longenum}
            \end{longenum}
            \item Software
            \item Algoritmo
            \begin{longenum}
                \item Definición
                \item Características
                \item Representación
                \begin{longenum}
                    \item Ordinograma
                    \item Pseudocódigo
                \end{longenum}
                \item Cualidades deseables
                \item Computabilidad
                \item Corrección
                \item Complejidad
            \end{longenum}
            \item Programa
            \item Lenguaje de programación
        \end{longenum}
        \item Paradigmas de programación
        \begin{longenum}
            \item Definición
            \item Imperativo
            \begin{longenum}
                \item Estructurado
                \item Orientado a objetos
            \end{longenum}
            \item Declarativo
            \begin{longenum}
                \item Funcional
                \item Lógico
                \item De bases de datos
            \end{longenum}
        \end{longenum}
        \item Lenguajes de programación
        \begin{longenum}
            \item Definición
            \begin{longenum}
                \item Sintaxis
                \begin{longenum}
                    \item Notación EBNF
                \end{longenum}
                \item Semántica estática
                \item Semántica dinámica
            \end{longenum}
            \item Evolución histórica
            \item Clasificación
            \begin{longenum}
                \item Por nivel
                \item Por generación
                \item Por propósito
                \item Por paradigma
            \end{longenum}
        \end{longenum}
        \item Traductores
        \begin{longenum}
            \item Definición
            \item Compiladores
            \begin{longenum}
                \item Ensambladores
            \end{longenum}
            \item Intérpretes
            \begin{longenum}
                \item Interactivos (\textit{REPL})
            \end{longenum}
        \end{longenum}
        \item Resolución de problemas mediante programación
        \begin{longenum}
            \item Especificación
            \item Análisis del problema
            \item Diseño del algoritmo
            \item Verificación
            \item Estudio de la eficiencia
            \item Codificación
            \item Traducción y ejecución
            \item Pruebas
            \item Depuración
            \item Documentación
            \item Mantenimiento
            \item Ingeniería del software
        \end{longenum}
        \item Entornos integrados de desarrollo
        \begin{longenum}
            \item Definición
            \item Editores de textos
            \item Editores vs. IDE
            \item Visual Studio Code
        \end{longenum}
    \end{longenum}
    \item \textbf{\textsc{Programación funcional I}} \ce{1a}\ \ce{1b}\ \ce{1c}\ \ce{1e}\ \ce{1f}\ \ce{1g}\ \ce{1i}\ \ce{3f}\ \ce{3g}\ \ev1\ \ra1\ \ra3\ \ra6\ (est: 2019\==09\==24)
    \begin{longenum}
        \item El lenguaje de programación Python
        \begin{longenum}
            \item Historia
            \item Características principales
        \end{longenum}
        \item Modelo de ejecución
        \begin{longenum}
            \item Modelo de sustitución
        \end{longenum}
        \item Expresiones
        \begin{longenum}
            \item Concepto
            \item Evaluación de expresiones
            \begin{longenum}
                \item Valores, expresión canónica y forma normal
                \item Formas normales y evaluación
                \item Transparencia referencial
            \end{longenum}
            \item Literales
            \item Operaciones, operadores y operandos
            \begin{longenum}
                \item Aridad de operadores
                \item Paréntesis
                \item Prioridad de operadores
                \item Asociatividad de operadores
            \end{longenum}
            \item Funciones y métodos
            \begin{longenum}
                \item Funciones
                \item Igualdad de funciones
                \item Funciones con varios argumentos
                \item Composición de funciones
                \item Métodos
            \end{longenum}
            \item Tipos de datos
            \begin{longenum}
                \item Concepto
                \item \texttt{type}
                \item Sistemas de tipos
                \item Tipado fuerte vs. débil
                \item Errores de tipos
                \item Tipos de datos básicos
                \begin{longenum}
                    \item Números
                    \item Cadenas
                \end{longenum}
                \item Conversión de tipos
            \end{longenum}
            \item Operaciones predefinidas
            \begin{longenum}
                \item Operadores predefinidos
                \begin{longenum}
                    \item Operadores aritméticos
                    \item Operadores de cadenas
                \end{longenum}
                \item Funciones predefinidas
                \begin{longenum}
                    \item Funciones matemáticas
                \end{longenum}
                \item Métodos predefinidos
            \end{longenum}
        \end{longenum}
        \item Álgebra de Boole
        \begin{longenum}
            \item El tipo de dato \textit{booleano}
            \item Operadores relacionales
            \item Operadores lógicos
            \item Axiomas
            \begin{longenum}
                \item Traducción a Python
            \end{longenum}
            \item Teoremas fundamentales
            \begin{longenum}
                \item Traducción a Python
            \end{longenum}
            \item El operador ternario
        \end{longenum}
        \item Definiciones
        \begin{longenum}
            \item Identificadores y ligaduras (\textit{binding})
            \begin{longenum}
                \item Reglas léxicas
                \item Constantes
            \end{longenum}
            \item Evaluación de expresiones con ligaduras
            \item Marcos (\textit{frames})
            \item Entorno (\textit{environment})
            \item Tipado estático vs. dinámico
            \item \textit{Scripts}
            \item Ámbito de una ligadura
        \end{longenum}
        \item Documentación interna
        \begin{longenum}
            \item Identificadores significativos
            \item Comentarios
        \end{longenum}
    \end{longenum}
    \item \textbf{\textsc{Programación funcional II}} \ce{1a}\ \ce{1b}\ \ce{1c}\ \ce{3f}\ \ce{3g}\ \ev1\ \ra1\ \ra3\ \ra6\ (est: 2019\==10\==01)
    \begin{longenum}
        \item Abstracciones funcionales
        \begin{longenum}
            \item Expresiones lambda
            \begin{longenum}
                \item Parámetros y argumentos
                \item Aplicación funcional
                \begin{longenum}
                    \item Llamadas a funciones
                    \item Evaluación de una aplicación funcional
                \end{longenum}
                \item Variables ligadas y libres
                \item Ámbitos
                \begin{longenum}
                    \item Ámbito de una variable ligada
                    \item Ámbitos, marcos y entornos
                    \item Variables \textit{sombreadas}
                    \item Renombrado de parámetros
                    \item Expresiones lambda y entornos
                    \item Evaluación de expresiones lambda con entornos
                \end{longenum}
            \end{longenum}
            \item Estrategias de evaluación
            \begin{longenum}
                \item Orden de evaluación
                \begin{longenum}
                    \item Orden aplicativo
                    \item Orden normal
                \end{longenum}
                \item Evaluación estricta y no estricta
            \end{longenum}
            \item Composición de funciones
        \end{longenum}
        \item Computabilidad
        \begin{longenum}
            \item Funciones y procesos
            \item Funciones recursivas
            \begin{longenum}
                \item Definición
                \item Casos base y casos recursivos
                \item El factorial
                \item Recursividad lineal
                \begin{longenum}
                    \item Procesos lineales recursivos
                    \item Procesos lineales iterativos
                \end{longenum}
                \item Recursividad en árbol
            \end{longenum}
            \item Un lenguaje Turing-completo
        \end{longenum}
        \item Tipos de datos recursivos
        \begin{longenum}
            \item Cadenas
            \item Listas
        \end{longenum}
        \item Funciones de orden superior
        \begin{longenum}
            \item Concepto
            \item \texttt{map}
            \item \texttt{filter}
            \item \texttt{reduce}
        \end{longenum}
    \end{longenum}
    \item \textbf{\textsc{Programación imperativa}} \ce{1a}\ \ce{1b}\ \ce{1c}\ \ce{3f}\ \ce{3g}\ \ev1\ \ra1\ \ra3\ \ra6\ (est: 2019\==10\==08)
    \begin{longenum}
        \item Modelo de ejecución
        \begin{longenum}
            \item Máquina de estados
            \item Secuencia de instrucciones
        \end{longenum}
        \item Asignación destructiva
        \begin{longenum}
            \item Variables
            \item Estado
            \item Sentencia de asignación
            \item Evaluación de expresiones con variables
            \item Constantes
        \end{longenum}
        \item Mutabilidad
        \begin{longenum}
            \item Tipos mutables e inmutables
            \begin{longenum}
                \item Inmutables
                \item Mutables
            \end{longenum}
            \item Alias de variables
            \begin{longenum}
                \item \texttt{id}
                \item \texttt{is}
            \end{longenum}
        \end{longenum}
        \item Efectos laterales
        \begin{longenum}
            \item Concepto
            \item Transparencia referencial
            \item Entrada y salida por consola
            \begin{longenum}
                \item \texttt{print}
                \begin{longenum}
                    \item El valor \texttt{None}
                \end{longenum}
                \item \texttt{input}
            \end{longenum}
        \end{longenum}
        \item Saltos
        \begin{longenum}
            \item Incondicionales
            \item Condicionales
        \end{longenum}
    \end{longenum}
    \item \textbf{\textsc{Programación estructurada}} \ce{1a}\ \ce{1b}\ \ce{1c}\ \ce{3a}\ \ce{3f}\ \ce{3g}\ \ev1\ \ra1\ \ra3\ \ra6\ (est: 2019\==10\==15)
    \begin{longenum}
        \item Funciones definidas por el usuario
        \begin{longenum}
            \item Definición de funciones con nombre
            \item Paso de argumentos
            \begin{longenum}
                \item Por valor
                \item Por referencia
                \item Por asignación
            \end{longenum}
            \item La sentencia \texttt{return}
            \item Ámbito de variables
            \begin{longenum}
                \item Variables globales
                \begin{longenum}
                    \item \texttt{global}
                    \item Efectos laterales
                \end{longenum}
                \item Variables locales
            \end{longenum}
            \item Declaraciones de tipos
            \begin{longenum}
                \item Declaraciones de tipo de argumento
                \item Declaraciones de tipo de devolución
            \end{longenum}
            \item Funciones locales a funciones
            \begin{longenum}
                \item \texttt{nonlocal}
            \end{longenum}
            \item \textit{Docstrings}
        \end{longenum}
        \item Teorema de Böhm-Jacopini
        \item Estructuras básicas de control
        \begin{longenum}
            \item Concepto de estructura
            \item Secuencia
            \item Selección
            \item Iteración
        \end{longenum}
        \item Metodología de la programación estructurada
        \begin{longenum}
            \item Recursos abstractos
            \item Diseño descendente
            \item Refinamiento sucesivo
        \end{longenum}
        \item Captura de excepciones
    \end{longenum}
    \item \textbf{\textsc{Tipos de datos}} \ce{1d}\ \ce{1h}\ \ce{3f}\ \ce{3g}\ \ce{6g}\ \ev1\ \ra1\ \ra3\ \ra6\ (est: 2019\==10\==22)
    \begin{longenum}
        \item Tipos básicos
        \begin{longenum}
            \item Lógicos (\texttt{bool})
            \begin{longenum}
                \item Operadores lógicos
                \item Operadores de comparación
            \end{longenum}
            \item Numéricos
            \begin{longenum}
                \item Enteros (\texttt{int})
                \item Números en coma flotante (\texttt{float})
                \item Operadores
                \begin{longenum}
                    \item Operadores aritméticos
                    \item Operadores de asignación compuesta
                \end{longenum}
                \item Funciones
                \item Métodos
            \end{longenum}
            \item Nulo (\texttt{None})
        \end{longenum}
        \item Tipos compuestos
        \begin{longenum}
            \item Secuencias
            \begin{longenum}
                \item Operaciones comunes
                \item Cadenas (\texttt{str})
                \begin{longenum}
                    \item Operadores
                    \begin{longenum}
                        \item Concatenación
                        \item Repetición
                        \item Indexación
                        \item \textit{Slicing}
                    \end{longenum}
                    \item Funciones
                    \item Métodos
                    \item Expresiones regulares
                \end{longenum}
                \item Listas
                \item Tuplas
                \item Rangos
            \end{longenum}
            \item Conjuntos (\texttt{set} y \texttt{frozenset})
            \item Diccionarios (\texttt{dict})
        \end{longenum}
        \item Manipulación de tipos
        \begin{longenum}
            \item Comprobaciones
            \begin{longenum}
                \item De tipos
                \begin{longenum}
                    \item \texttt{type()}
                \end{longenum}
                \item De valores
                \begin{longenum}
                    \item \texttt{str.isnumeric()}
                    \item \texttt{str.isdigit()}
                \end{longenum}
            \end{longenum}
            \item Conversiones
            \begin{longenum}
                \item Conversión explícita vs. implícita
                \item Conversión a \texttt{bool}
                \item Conversión a \texttt{int}
                \item Conversión a \texttt{float}
                \item Conversión a \texttt{str}
            \end{longenum}
            \item Comparaciones
            \begin{longenum}
                \item Operadores de comparación
            \end{longenum}
        \end{longenum}
    \end{longenum}
    \item \textbf{\textsc{Metodología de la programación}} \ce{1a}\ \ce{1b}\ \ce{1c}\ \ce{3f}\ \ce{3g}\ \ev1\ \ra1\ \ra3\ \ra6\ (est: 2019\==10\==29)
    \begin{longenum}
        \item Ciclo de vida
        \item Especificación e implementación
        \item Verificación y validación de programas
        \begin{longenum}
            \item Demostraciones por inducción
        \end{longenum}
        \item Programación funcional
        \begin{longenum}
            \item Especificaciones formales
            \begin{longenum}
                \item Como cálculo
            \end{longenum}
            \item Derivación de programas
            \begin{longenum}
                \item Diseño recursivo
                \begin{longenum}
                    \item Recursividad final
                    \item Técnicas de inmersión \opcional\
                \end{longenum}
            \end{longenum}
        \end{longenum}
        \item Programación imperativa
        \begin{longenum}
            \item Especificaciones formales
            \begin{longenum}
                \item Como modificación de estados
            \end{longenum}
            \item Derivación de programas
            \begin{longenum}
                \item Diseño iterativo
                \begin{longenum}
                    \item Invariante de un bucle
                    \item Transformación de recursividad final a iterativo
                \end{longenum}
            \end{longenum}
        \end{longenum}
        \item El lenguaje Dafny \opcional\
    \end{longenum}
    \item \textbf{\textsc{Programación modular I}} \ce{1a}\ \ce{1b}\ \ce{1c}\ \ce{3f}\ \ce{3g}\ \ev1\ \ra1\ \ra3\ \ra6\ (est: 2019\==11\==05)
    \begin{longenum}
        \item Introducción
        \begin{longenum}
            \item Descomposición de problemas
        \end{longenum}
        \item Partes de un módulo
        \begin{longenum}
            \item Interfaz
            \item Implementación
            \item Documentación interna
        \end{longenum}
        \item Importación de módulos
        \item Paquetes
        \item Criterios de descomposición modular
        \begin{longenum}
            \item Abstracción
            \item Ocultación de información
            \item Independencia funcional
            \begin{longenum}
                \item Cohesión
                \item Acoplamiento
            \end{longenum}
            \item Reusabilidad
        \end{longenum}
        \item Diagramas de estructura
    \end{longenum}
    \item \textbf{\textsc{Entrada y salida de información}} \ce{1a}\ \ce{1b}\ \ce{1c}\ \ce{3f}\ \ce{3g}\ \ce{5a}\ \ce{5b}\ \ce{5c}\ \ce{5d}\ \ce{5e}\ \ce{6h}\ \ce{6i}\ \ev1\ \ra1\ \ra3\ \ra5\ \ra6\ (est: 2019\==11\==12)
    \begin{longenum}
        \item Formateado de cadenas
        \item La consola
        \begin{longenum}
            \item Entrada desde teclado
            \item Salida a pantalla
        \end{longenum}
        \item Archivos de datos
        \begin{longenum}
            \item Registros
            \item Apertura y cierre de archivos
            \item Gestores de contexto
            \item Modos de acceso
            \item Lectura y escritura de información en archivos
        \end{longenum}
        \item Manipulación de archivos XML
        \item Serialización de objetos
        \item Sistemas de archivos
        \begin{longenum}
            \item Manipulación de los sistemas de archivos
            \item Creación y eliminación de archivos y directorios
        \end{longenum}
    \end{longenum}
    \item \textbf{\textsc{Calidad I}} \ce{1a}\ \ce{1b}\ \ce{1c}\ \ce{3f}\ \ce{3g}\ \ev1\ \ra1\ \ra3\ (est: 2019\==11\==19)
    \begin{longenum}
        \item Depuración
        \begin{longenum}
            \item \texttt{print()}
            \item Depuración en el IDE
        \end{longenum}
        \item Pruebas
        \begin{longenum}
            \item Enfoques de pruebas
            \begin{longenum}
                \item Pruebas de caja blanca
                \item Pruebas de caja negra
            \end{longenum}
            \item Estrategias de pruebas
            \begin{longenum}
                \item Unitarias
                \item Funcionales
                \item De aceptación
            \end{longenum}
            \item \texttt{doctest}
            \item \texttt{pytest}
            \item Desarrollo conducido por pruebas
            \begin{longenum}
                \item Ciclo de desarrollo
                \item Ventajas
            \end{longenum}
        \end{longenum}
        \item Documentación
        \begin{longenum}
            \item Interna
            \begin{longenum}
                \item Comentarios
                \item \textit{Docstrings}
                \item Reglas de estilo
            \end{longenum}
            \item Externa
            \begin{longenum}
                \item \texttt{pydoc}
            \end{longenum}
        \end{longenum}
    \end{longenum}
    \item \textbf{\textsc{Complejidad algorítmica}} \ev1\ \opcional\ (est: 2019\==11\==26)
    \begin{longenum}
        \item Introducción
        \item Principio de invarianza
        \item La notación asintótica \textit{O(f(n))}
        \item Órdenes de complejidad
        \item Operaciones entre órdenes de complejidad
        \begin{longenum}
            \item Regla de la suma
            \item Regla del producto
        \end{longenum}
        \item Reglas prácticas para el cálculo de la eficiencia
        \item Resolución de recurrencias
        \begin{longenum}
            \item Reducción de problemas mediante sustracción
            \item Reducción de problemas mediante división
        \end{longenum}
    \end{longenum}
    \item \textbf{\textsc{Introducción al lenguaje Java}} \ce{1a}\ \ce{1b}\ \ce{1c}\ \ce{3f}\ \ce{3g}\ \ev2\ \ra1\ \ra3\ (est: 2020\==01\==15)
    \begin{longenum}
        \item Introducción
        \item Compilación vs. interpretación
        \begin{longenum}
            \item Máquinas reales vs. virtuales
            \item Código objeto, \textit{bytecode} y archivos \texttt{.class}
            \item La plataforma Java
            \begin{longenum}
                \item La máquina virtual de Java (JVM)
                \item La API de Java
            \end{longenum}
            \item El entorno de ejecución de Java (JRE)
            \begin{longenum}
                \item El intérprete \texttt{java}
            \end{longenum}
            \item Las herramientas de desarrollo de Java (JDK)
            \begin{longenum}
                \item El compilador \texttt{javac}
                \item El intérprete interactivo \texttt{jshell}
            \end{longenum}
        \end{longenum}
        \item Características de Java
        \item Tipado estático vs. dinámico
        \item Tipos y valores en Java
        \begin{longenum}
            \item Tipos  primitivos
            \begin{longenum}
                \item Integrales
                \item De coma flotante
                \item Booleanos
                \item Conversiones entre datos primitivos
                \begin{longenum}
                    \item \textit{Casting}
                    \item De ampliación (\textit{widening})
                    \item De restricción (\textit{narrowing})
                \end{longenum}
                \item Promociones numéricas
            \end{longenum}
            \item Tipos por referencia
        \end{longenum}
        \item Variables en Java
        \begin{longenum}
            \item Variables de tipos primitivos
            \item Variables de tipos por referencia
            \item Declaraciones de variables
        \end{longenum}
    \end{longenum}
    \item \textbf{\textsc{Programación orientada a objetos}} \ce{1a}\ \ce{1b}\ \ce{1c}\ \ce{2a}\ \ce{2b}\ \ce{2c}\ \ce{2d}\ \ce{2f}\ \ce{2h}\ \ce{2i}\ \ce{3f}\ \ce{3g}\ \ce{6a}\ \ev2\ \ra1\ \ra2\ \ra3\ \ra6\ (est: 2020\==01\==22)
    \begin{longenum}
        \item Introducción
        \begin{longenum}
            \item Perspectiva histórica
            \item Lenguajes orientados a objetos
        \end{longenum}
        \item Conceptos básicos
        \begin{longenum}
            \item Clase
            \item Objeto
            \begin{longenum}
                \item La antisimetría dato-objeto
            \end{longenum}
            \item Identidad
            \item Estado
            \item Propiedad
            \item Paso de mensajes
            \item Método
            \item Encapsulación
            \item Herencia
            \item Polimorfismo
        \end{longenum}
        \item Uso básico de objetos
        \begin{longenum}
            \item Instanciación
            \begin{longenum}
                \item \texttt{new}
                \item \texttt{instanceof}
            \end{longenum}
            \item Propiedades
            \begin{longenum}
                \item Acceso y modificación
            \end{longenum}
            \item Referencias
            \item Clonación de objetos
            \item Comparación de objetos
            \item Destrucción de objetos
            \begin{longenum}
                \item Recolección de basura
            \end{longenum}
            \item Métodos
            \item Constantes
        \end{longenum}
        \item Clases básicas
        \begin{longenum}
            \item Cadenas
            \begin{longenum}
                \item Inmutables
                \begin{longenum}
                    \item \texttt{String}
                \end{longenum}
                \item Mutables
                \begin{longenum}
                    \item \texttt{StringBuffer}
                    \item \texttt{StringBuilder}
                    \item \texttt{StringTokenizer}
                \end{longenum}
                \item Conversión a \texttt{String}
            \end{longenum}
            \item \textit{Arrays}
            \item Clases \textit{wrapper}
            \begin{longenum}
                \item Conversiones de empaquetado/desempaquetado (\textit{boxing}/\textit{unboxing})
            \end{longenum}
        \end{longenum}
        \item Lenguaje UML
        \begin{longenum}
            \item Diagramas de clases
            \item Diagramas de objetos
            \item Diagramas de secuencia
        \end{longenum}
    \end{longenum}
    \item \textbf{\textsc{Diseño de clases}} \ce{1a}\ \ce{1b}\ \ce{1c}\ \ce{2e}\ \ce{3f}\ \ce{3g}\ \ce{4a}\ \ce{4b}\ \ce{4c}\ \ce{4d}\ \ce{4e}\ \ce{4f}\ \ce{4h}\ \ev2\ \ra1\ \ra2\ \ra3\ \ra4\ (est: 2020\==01\==29)
    \begin{longenum}
        \item Encapsulación
        \item Propiedades
        \begin{longenum}
            \item Visibilidad
            \begin{longenum}
                \item Pública
                \item Privada
            \end{longenum}
        \end{longenum}
        \item Métodos
        \begin{longenum}
            \item Visibilidad
            \begin{longenum}
                \item Pública
                \item Privada
            \end{longenum}
            \item Referencia \texttt{this}
            \item Sobrecarga
            \item Constructores y destructores
            \item Accesores y mutadores
        \end{longenum}
        \item Constantes
        \item Miembros estáticos
        \begin{longenum}
            \item Constantes
            \item Métodos estáticos
            \item Propiedades estáticas
        \end{longenum}
        \item El primer programa Java
    \end{longenum}
    \item \textbf{\textsc{Composición, herencia y poliformismo}} \ce{1a}\ \ce{1b}\ \ce{1c}\ \ce{3f}\ \ce{3g}\ \ce{4g}\ \ce{7a}\ \ce{7b}\ \ce{7c}\ \ce{7d}\ \ce{7e}\ \ce{7f}\ \ce{7g}\ \ce{7h}\ \ev2\ \ra1\ \ra3\ \ra4\ \ra7\ (est: 2020\==02\==05)
    \begin{longenum}
        \item Composición de clases
        \item Herencia
        \begin{longenum}
            \item Concepto de herencia
            \item Modos
            \begin{longenum}
                \item Simple
                \item Múltiple
            \end{longenum}
            \item Superclases y subclases
            \item La clase \texttt{Object}
            \item Visibilidad protegida
            \item Utilización de clases heredadas
            \item Constructores y herencia
            \item \texttt{super}
            \item Restricciones
            \begin{longenum}
                \item Clases y métodos abstractos
                \item Clases y métodos finales
            \end{longenum}
        \end{longenum}
        \item Polimorfismo
        \begin{longenum}
            \item El principio de sustitución de Liskov
            \item Conversiones entre tipos referencia
            \item Sobreescritura de métodos
            \begin{longenum}
                \item Covarianza en el tipo de retorno
                \item Invarianza en el tipo de los argumentos
            \end{longenum}
            \item Sobreescritura de constructores
        \end{longenum}
        \item Herencia vs. composición
    \end{longenum}
    \item \textbf{\textsc{Abstracción de datos}} \ce{1a}\ \ce{1b}\ \ce{1c}\ \ce{3f}\ \ce{3g}\ \ev2\ \ra1\ \ra3\ \ra6\ (est: 2020\==02\==12)
    \begin{longenum}
        \item Tipos abstractos de datos
        \begin{longenum}
            \item Concepto, terminología y ejemplos
            \item Programación con tipos abstractos de datos
            \begin{longenum}
                \item Modularidad
                \item Refinamientos sucesivos
                \item Programación a gran escala
                \item Programación genérica
            \end{longenum}
        \end{longenum}
        \item Especificación
        \begin{longenum}
            \item Especificaciones algebraicas
            \begin{longenum}
                \item Signatura de un TAD
                \begin{longenum}
                    \item Géneros
                    \item Operaciones
                    \begin{longenum}
                        \item Constructoras
                        \item Selectoras
                    \end{longenum}
                \end{longenum}
                \item Términos
                \item Ecuaciones
            \end{longenum}
            \item Construcción de especificaciones
            \item Verificación con especificaciones algebraicas
        \end{longenum}
        \item Implementación
        \begin{longenum}
            \item Pilas
            \item Colas
            \item Listas
        \end{longenum}
    \end{longenum}
    \item \textbf{\textsc{Programación modular II}} \ce{1a}\ \ce{1b}\ \ce{1c}\ \ce{3f}\ \ce{3g}\ \ce{4i}\ \ce{4j}\ \ev2\ \ra1\ \ra3\ \ra4\ (est: 2020\==02\==19)
    \begin{longenum}
        \item Las clases como módulos
        \begin{longenum}
            \item Interfaz de una clase
            \item Métodos \textit{getter} y \textit{setter}
        \end{longenum}
        \item Interfaces
        \begin{longenum}
            \item Definición de interfaces
            \item Implementación de interfaces
            \item Las interfaces como tipos
            \item Conversiones entre interfaces
            \item Métodos predeterminados
        \end{longenum}
        \item Paquetes y módulos
    \end{longenum}
    \item \textbf{\textsc{Programación genérica}} \ce{1a}\ \ce{1b}\ \ce{1c}\ \ce{3f}\ \ce{3g}\ \ce{6f}\ \ev2\ \ra1\ \ra3\ \ra6\ (est: 2020\==02\==26)
    \begin{longenum}
        \item Tipos genéricos
        \begin{longenum}
            \item Parámetros de tipo
            \item Argumentos de tipo
            \item Tipos crudos
        \end{longenum}
        \item Métodos genéricos
        \item Subtipos
        \begin{longenum}
            \item Parámetros de tipo acotados
            \item Clases genéricas, herencia y subtipos
            \begin{longenum}
                \item Covarianza
                \item Contravarianza
                \item Invarianza
            \end{longenum}
        \end{longenum}
        \item Inferencia de tipos
        \item Comodines
        \item Borrado de tipos
        \item Limitaciones
    \end{longenum}
    \item \textbf{\textsc{Estructuras de datos lineales}} \ce{1a}\ \ce{1b}\ \ce{1c}\ \ce{3f}\ \ce{3g}\ \ce{6a}\ \ev2\ \ra1\ \ra3\ \ra6\ (est: 2020\==03\==04)
    \begin{longenum}
        \item Acceso directo
        \begin{longenum}
            \item \textit{Arrays}
            \item \texttt{java.util.Arrays}
            \begin{longenum}
                \item Comparación de \textit{arrays}
                \begin{longenum}
                    \item \texttt{Arrays.equals()}
                    \item \texttt{Arrays.deepEquals()}
                \end{longenum}
            \end{longenum}
        \end{longenum}
        \item Acceso secuencial
        \begin{longenum}
            \item Listas
            \begin{longenum}
                \item Enlazadas
                \item Doblemente enlazadas
            \end{longenum}
            \item Pilas
            \item Colas
        \end{longenum}
    \end{longenum}
    \item \textbf{\textsc{Ordenación y búsqueda}} \ce{1a}\ \ce{1b}\ \ce{1c}\ \ce{3f}\ \ce{3g}\ \ev2\ \ra1\ \ra3\ \ra6\ (est: 2020\==03\==11)
    \begin{longenum}
        \item Algoritmos de búsqueda
        \begin{longenum}
            \item Búsqueda secuencial
            \item Búsqueda dicotómica
        \end{longenum}
        \item Algoritmos de ordenación
        \begin{longenum}
            \item Inserción directa
            \item Selección directa
            \item Burbuja
            \item \textit{Quicksort}
            \item \textit{Mergesort}
        \end{longenum}
        \item Tablas \textit{Hash}
    \end{longenum}
    \item \textbf{\textsc{Estructuras de datos no lineales}} \ce{1a}\ \ce{1b}\ \ce{1c}\ \ce{3f}\ \ce{3g}\ \ev2\ \ra1\ \ra3\ \ra6\ (est: 2020\==03\==18)
    \begin{longenum}
        \item Árboles
        \begin{longenum}
            \item Binarios
            \begin{longenum}
                \item Recorridos
                \begin{longenum}
                    \item Preorden
                    \item Inorden
                    \item Postorden
                \end{longenum}
            \end{longenum}
            \item De búsqueda
            \item Montículos
            \begin{longenum}
                \item Algoritmo de ordenación
            \end{longenum}
            \item Generales
            \begin{longenum}
                \item Recorrido en profundidad
                \item Recorrido en anchura
            \end{longenum}
        \end{longenum}
        \item Grafos
        \begin{longenum}
            \item Algoritmo de Dijkstra
            \item Algoritmo de Floyd
        \end{longenum}
    \end{longenum}
    \item \textbf{\textsc{Control de excepciones}} \ce{1a}\ \ce{1b}\ \ce{1c}\ \ce{3d}\ \ce{3f}\ \ce{3g}\ \ev2\ \ra1\ \ra3\ (est: 2020\==03\==25)
    \begin{longenum}
        \item Errores y excepciones
        \item El requisito «\textit{captura o especifica}»
        \begin{longenum}
            \item Tipos de excepciones
        \end{longenum}
        \item Captura y manejo de excepciones
        \begin{longenum}
            \item Bloque \texttt{try}
            \item Bloques \texttt{catch}
            \item Bloque \texttt{finally}
        \end{longenum}
        \item Excepciones y signaturas
        \item Lanzamiento de excepciones
        \begin{longenum}
            \item Excepciones encadenadas
            \item Creación de clases de excepción
        \end{longenum}
        \item Excepciones no chequeadas
        \item Ventajas de las excepciones
    \end{longenum}
    \item \textbf{\textsc{Java Collections Framework}} \ce{1a}\ \ce{1b}\ \ce{1c}\ \ce{3f}\ \ce{3g}\ \ce{6a}\ \ce{6b}\ \ce{6c}\ \ce{6d}\ \ce{6e}\ \ce{6f}\ \ev3\ \ra1\ \ra3\ \ra6\ (est: 2020\==04\==13)
    \begin{longenum}
        \item Colecciones y \textit{arrays}
        \item Arquitectura
        \item Tipos de colecciones
        \begin{longenum}
            \item Listas ordenadas
            \item Diccionarios
            \item Conjuntos
        \end{longenum}
        \item Listas
        \begin{longenum}
            \item \texttt{java.util.List}
            \begin{longenum}
                \item \texttt{java.util.ArrayList}
                \item \texttt{java.util.LinkedList}
                \item \texttt{java.util.Stack}
            \end{longenum}
        \end{longenum}
        \item Colas
        \begin{longenum}
            \item \texttt{java.util.Queue}
            \begin{longenum}
                \item \texttt{java.util.ArrayDeque}
                \item \texttt{java.util.PriorityQueue}
                \item \texttt{java.util.LinkedList}
            \end{longenum}
            \item \texttt{java.util.Deque}
            \begin{longenum}
                \item \texttt{java.util.ArrayDeque}
                \item \texttt{java.util.LinkedList}
            \end{longenum}
        \end{longenum}
        \item Conjuntos
        \begin{longenum}
            \item \texttt{java.util.Set}
            \begin{longenum}
                \item \texttt{java.util.HashSet}
                \item \texttt{java.util.LinkedHashSet}
                \item \texttt{java.util.TreeSet}
            \end{longenum}
            \item \texttt{java.util.SortedSet}
            \begin{longenum}
                \item \texttt{java.util.TreeSet}
            \end{longenum}
            \item \texttt{java.util.NavigableSet}
            \begin{longenum}
                \item \texttt{java.util.TreeSet}
            \end{longenum}
        \end{longenum}
        \item Diccionarios
        \begin{longenum}
            \item \texttt{java.util.Map}
            \begin{longenum}
                \item \texttt{java.util.HashMap}
                \item \texttt{java.util.LinkedHashMap}
                \item \texttt{java.util.TreeMap}
            \end{longenum}
            \item \texttt{java.util.SortedMap}
            \begin{longenum}
                \item \texttt{java.util.TreeMap}
            \end{longenum}
            \item \texttt{java.util.NavigableMap}
            \begin{longenum}
                \item \texttt{java.util.TreeMap}
            \end{longenum}
        \end{longenum}
    \end{longenum}
    \item \textbf{\textsc{Principios y patrones de diseño}} \ce{1a}\ \ce{1b}\ \ce{1c}\ \ce{3f}\ \ce{3g}\ \ev3\ \ra1\ \ra3\ (est: 2020\==04\==20)
    \begin{longenum}
        \item Principios de diseño
        \begin{longenum}
            \item Encapsulación y ocultación de información
            \item Diseño orientado a interfaces
            \item Principios \textit{SOLID}
            \begin{longenum}
                \item SRP: Principio de responsabilidad única
                \item OCP: Principio de abierto/cerrado
                \item LSP: Principio de sustitución de Liskov
                \item ISP: Principio de segregación de la interfaz
                \item DIP: Principio de inversión de dependencias
            \end{longenum}
            \item Principio del Menor Conocimiento (o Ley de Demeter)
        \end{longenum}
        \item Patrones de diseño
        \begin{longenum}
            \item De creación
            \item Estructurales
            \item De comportamiento
        \end{longenum}
    \end{longenum}
    \item \textbf{\textsc{Calidad II}} \ce{1a}\ \ce{1b}\ \ce{1c}\ \ce{3f}\ \ce{3g}\ \ev3\ \ra1\ \ra3\ (est: 2020\==04\==27)
    \begin{longenum}
        \item Pruebas automáticas
        \begin{longenum}
            \item JUnit
        \end{longenum}
        \item Documentación
        \begin{longenum}
            \item Interna
            \begin{longenum}
                \item Reglas de estilo
                \item Google Java Format
            \end{longenum}
            \item Externa
            \begin{longenum}
                \item Javadoc
            \end{longenum}
        \end{longenum}
    \end{longenum}
    \item \textbf{\textsc{Gestión de bases de datos relacionales}} \ce{1a}\ \ce{1b}\ \ce{1c}\ \ce{3f}\ \ce{3g}\ \ce{8a}\ \ce{8b}\ \ce{8c}\ \ce{8d}\ \ce{8e}\ \ce{8f}\ \ce{8g}\ \ce{8h}\ \ce{9a}\ \ce{9b}\ \ce{9c}\ \ce{9d}\ \ce{9e}\ \ce{9f}\ \ce{9g}\ \ev3\ \ra1\ \ra3\ \ra6\ \ra8\ \ra9\ (est: 2020\==05\==04)
    \begin{longenum}
        \item Controlador JDBC
        \begin{longenum}
            \item Instalación
            \item \texttt{CLASSPATH}
            \item Carga
        \end{longenum}
        \item Establecimiento de conexiones
        \item Recuperación de información
        \begin{longenum}
            \item Ejecución de consultas
            \item Selección de registros
            \item Uso de parámetros
        \end{longenum}
        \item Manipulación de la información
        \begin{longenum}
            \item Altas, bajas y modificaciones
        \end{longenum}
    \end{longenum}
    \item \textbf{\textsc{Programación de interfaces gráficas de usuario}} \ce{1a}\ \ce{1b}\ \ce{1c}\ \ce{3f}\ \ce{3g}\ \ce{5f}\ \ce{5g}\ \ce{5h}\ \ev3\ \opcional\ \ra1\ \ra3\ \ra5\ \ra6\ (est: 2020\==05\==11)
    \begin{longenum}
        \item JFC y Swing
        \item Componentes de Swing
        \item Contenedores de nivel superior
        \begin{longenum}
            \item \texttt{JFrame}
            \item \texttt{JDialog}
            \item \texttt{JApplet}
        \end{longenum}
        \item \texttt{JComponent}
        \item Componentes de texto
        \begin{longenum}
            \item \texttt{JTextComponent}
        \end{longenum}
        \item Marcos
        \item Etiquetas
        \item Botones
        \item Arquitectura de modelos de Swing
    \end{longenum}
\end{longenum}
