\item [1] **CONTENIDOS**
\begin{longenum}
    \item **Introducción**
    \begin{longenum}
        \item Funcionamiento básico de un ordenador
        \begin{longenum}
            \item Elementos funcionales
            \item Ciclo de instrucción
            \item Representación de información
            \begin{longenum}
                \item Codificación interna
                \begin{longenum}
                    \item Sistema binario
                \end{longenum}
                \item Codificación externa
                \begin{longenum}
                    \item ASCII
                    \item Unicode
                \end{longenum}
            \end{longenum}
        \end{longenum}
        \item Resolución de problemas mediante programación
        \begin{longenum}
            \item Análisis del problema
            \item Especificación
            \item Diseño del algoritmo
            \item Codificación del algoritmo en forma de programa
        \end{longenum}
        \item Conceptos básicos
        \begin{longenum}
            \item Algoritmo
            \begin{longenum}
                \item Concepto
                \item Características
                \item Representación
                \begin{longenum}
                    \item Ordinograma
                    \item Pseudocódigo
                \end{longenum}
                \item Cualidades deseables
                \item Computabilidad
                \item Corrección
                \item Complejidad
            \end{longenum}
            \item Programa
            \item Lenguaje de programación
        \end{longenum}
        \item Evolución histórica
        \begin{longenum}
            \item Culturas de la programación
            \item Ingeniería del software
        \end{longenum}
        \item Paradigmas de programación
        \begin{longenum}
            \item Imperativo
            \begin{longenum}
                \item Estructurado
                \item Orientado a objetos
            \end{longenum}
            \item Declarativo
            \begin{longenum}
                \item Funcional
                \item Lógico
            \end{longenum}
        \end{longenum}
        \item Lenguajes de programación
        \begin{longenum}
            \item Definición
            \begin{longenum}
                \item Sintaxis
                \begin{longenum}
                    \item Notación EBNF
                \end{longenum}
                \item Semántica
            \end{longenum}
            \item Evolución histórica
            \item Clasificación
            \begin{longenum}
                \item Por nivel
                \item Por generación
                \item Por paradigma
            \end{longenum}
        \end{longenum}
        \item Traductores
        \begin{longenum}
            \item Compiladores
            \item Intérpretes
            \begin{longenum}
                \item Interactivos (*REPL*)
                \item Por lotes
            \end{longenum}
        \end{longenum}
        \item El lenguaje de programación PHP
        \begin{longenum}
            \item Historia
            \item Características principales
        \end{longenum}
        \item Entornos integrados de desarrollo
        \begin{longenum}
            \item Definición y tipos. Entornos comerciales y de Software libre.
            \item Instalación y descripción de entornos integrados de desarrollo.
            \item Creación de proyectos. Estructura y componentes.
        \end{longenum}
    \end{longenum}
    \item **Programación funcional I**
    \begin{longenum}
        \item Modelo de ejecución
        \begin{longenum}
            \item Evaluación de expresiones
            \item Modelo de sustitución / sistema de reescritura
        \end{longenum}
        \item Expresiones
        \begin{longenum}
            \item Valores, expresión canónica y forma normal
            \item Literales
            \item Operaciones, operadores y operandos
            \begin{longenum}
                \item [link: Precedencia y asociatividad de operadores|http://php.net/manual/es/language.operators.precedence.php]
            \end{longenum}
            \item Evaluación
            \begin{longenum}
                \item Orden de evaluación
                \begin{longenum}
                    \item Orden aplicativo
                    \item Orden normal
                \end{longenum}
            \end{longenum}
            \item Tipos de datos
            \begin{longenum}
                \item Concepto
                \begin{longenum}
                    \item Tipo de un valor
                    \item Tipo de una expresión
                \end{longenum}
                \item Tipos de datos básicos
                \begin{longenum}
                    \item Enteros
                    \begin{longenum}
                        \item Operadores aritméticos
                    \end{longenum}
                    \item Cadenas
                \end{longenum}
            \end{longenum}
            \item Algebraicas vs. algorítmicas
            \item Aritméticas
            \item Funciones predefinidas
            \item Constantes predefinidas
        \end{longenum}
        \item Álgebra de Boole
        \begin{longenum}
            \item El tipo de dato *booleano*
            \item Operadores relacionales
            \item Operadores lógicos
            \item Axiomas
            \item Propiedades
            \item El operador ternario
        \end{longenum}
        \item Sentencias
        \begin{longenum}
            \item Variables
            \begin{longenum}
                \item Identificadores
                \item Ligadura (*binding*)
                \item Estado
                \item Asignación no destructiva
                \item Tipado estático vs. dinámico
            \end{longenum}
            \item Evaluación de expresiones con variables
        \end{longenum}
        \item Autodocumentación
        \begin{longenum}
            \item Comentarios
            \item Reglas de estilo
        \end{longenum}
    \end{longenum}
    \item **Programación funcional II**
    \begin{longenum}
        \item Abstracciones funcionales
        \begin{longenum}
            \item Definición de funciones anónimas
            \item Parámetros y argumentos
            \item Paso de argumentos por valor
            \item Ámbito de las variables
            \item La sentencia `return`
        \end{longenum}
        \item Composición de funciones
        \item Computabilidad
        \begin{longenum}
            \item Funciones recursivas
            \item Un lenguaje Turing-completo
        \end{longenum}
        \item Tipos de datos compuestos
        \begin{longenum}
            \item Las cadenas como datos compuestos
            \item Los *arrays* como listas inmutables de elementos
        \end{longenum}
        \item Funciones de orden superior
        \begin{longenum}
            \item `array_map`
            \item `array_filter`
            \item `array_reduce`
            \item Funciones locales a funciones
            \item Funciones anónimas
        \end{longenum}
        \item *Scripts*
    \end{longenum}
    \item **Programación imperativa**
    \begin{longenum}
        \item Modelo de ejecución
        \begin{longenum}
            \item Máquina de estados
            \item Secuencia de instrucciones
        \end{longenum}
        \item Cambios de estado explícitos
        \begin{longenum}
            \item Celdas
            \item Asignación destructiva (o asignación múltiple)
            \item Asignación por referencia
        \end{longenum}
        \item Efectos laterales
        \begin{longenum}
            \item Transparencia referencial
            \item Entrada y salida por consola
            \begin{longenum}
                \item La sentencia `echo`
                \item Las funciones `var_dump()` y `print_r()`
                \item Las funciones `fgets()` y `fscanf()`
            \end{longenum}
        \end{longenum}
        \item Saltos
        \begin{longenum}
            \item Incondicionales: la sentencia `goto`
            \item Condicionales: la sentencia `if (...) goto`
            \item Implementación de bucles mediante saltos condicionales
        \end{longenum}
        \item Los *arrays* como estructura de datos mutable básica
        \begin{longenum}
            \item Creación, acceso y modificación
            \item Recorrido y búsqueda en un *array*
            \item *Arrays* multidimensionales
            \item Funciones de manejo de *arrays*
        \end{longenum}
    \end{longenum}
    \item **Programación estructurada**
    \begin{longenum}
        \item Teorema de Böhm-Jacopini
        \item Estructuras básicas de control
        \begin{longenum}
            \item Secuencia
            \item Selección
            \item Iteración
        \end{longenum}
        \item Recursos abstractos
        \item Diseño descendente
        \item Refinamiento sucesivo
    \end{longenum}
    \item **Tipos de datos**
    \begin{longenum}
        \item Introducción
        \item [link: Tipos básicos|http://php.net/manual/es/language.types.intro.php]
        \begin{longenum}
            \item [link: Lógicos (`bool`)|http://php.net/manual/es/language.types.boolean.php]
            \begin{longenum}
                \item [link: Operadores lógicos|http://php.net/manual/es/language.operators.logical.php]
            \end{longenum}
            \item Numéricos
            \begin{longenum}
                \item [link: Enteros (`int`)|http://php.net/manual/es/language.types.integer.php]
                \item [link: Números en coma flotante (`float`)|http://php.net/manual/es/language.types.float.php]
                \item Operadores
                \begin{longenum}
                    \item [link: Operadores aritméticos|http://php.net/manual/es/language.operators.arithmetic.php]
                    \item [link: Operadores de incremento/decremento|http://php.net/manual/es/language.operators.increment.php]
                \end{longenum}
            \end{longenum}
            \item [link: Cadenas (`string`)|http://php.net/manual/es/language.types.string.php]
            \begin{longenum}
                \item [link: Operadores de cadenas|http://php.net/manual/es/language.operators.string.php]
                \begin{longenum}
                    \item Concatenación
                    \item [link: Acceso y modificación por caracteres|http://php.net/manual/es/language.types.string.php#language.types.string.substr]
                    \item [link: Operador de incremento|http://php.net/manual/es/language.operators.increment.php]
                \end{longenum}
                \item [Funciones de manejo de cadenas](http://php.net/ref.strings)
                \item Expresiones regulares
                \item [Extensión *mbstring*](http://php.net/manual/en/book.mbstring.php)
            \end{longenum}
            \item [link: Nulo (`null`)|http://php.net/manual/es/language.types.null.php]
        \end{longenum}
        \item Tipos compuestos
        \begin{longenum}
            \item [*Arrays* asociativos](http://php.net/manual/es/language.types.array.php)
            \begin{longenum}
                \item [link: Operadores para arrays|http://php.net/manual/es/language.operators.array.php]
                \begin{longenum}
                    \item [link: Acceso, modificación y agregación|http://php.net/manual/es/language.types.array.php#language.types.array.syntax.modifying]
                \end{longenum}
                \item [link: [Funciones de manejo de arrays](http://php.net/manual/es/book.array.php)|http://php.net/manual/es/ref.array.php]
                \begin{longenum}
                    \item [link: Ordenación de arrays|http://php.net/manual/es/array.sorting.php]
                    \item `print_r()`
                    \item `'+'` vs. `array_merge()`
                    \item [`isset()` vs. `array_key_exists()`](http://php.net/manual/es/function.array-key-exists.php#107786)
                \end{longenum}
                \item [`foreach`](http://php.net/manual/es/control-structures.foreach.php)
                \item [link: Conversión a `array`|http://php.net/manual/es/language.types.array.php#language.types.array.casting]
                \item [link: *Ejemplo*: `$argv` en CLI|http://php.net/manual/es/reserved.variables.argv.php]
            \end{longenum}
            \item [link: Callables|http://php.net/manual/es/language.types.callable.php]
            \begin{longenum}
                \item [link: `call_user_func()`|http://php.net/manual/es/function.call-user-func.php]
                \item `array_map()` y `array_reduce()`
            \end{longenum}
            \item [link: Iterable|https://www.php.net/manual/es/language.types.iterable.php]
        \end{longenum}
        \item Manipulación de tipos
        \begin{longenum}
            \item Operadores de asignación compuesta
            \item Comprobaciones
            \begin{longenum}
                \item De tipos
                \begin{longenum}
                    \item [`gettype()`](http://php.net/manual/en/function.gettype.php)
                    \item [`is_*()`](http://php.net/manual/es/ref.var.php)
                \end{longenum}
                \item De valores
                \begin{longenum}
                    \item [`is_numeric()`](http://php.net/manual/es/function.is-numeric.php)
                    \item [`ctype_*()`](http://php.net/manual/es/book.ctype.php)
                \end{longenum}
            \end{longenum}
            \item [link: Conversiones de tipos|http://php.net/manual/es/language.types.type-juggling.php]
            \begin{longenum}
                \item [link: Conversión explícita (forzado o *casting*) vs. automática|http://php.net/manual/es/language.types.type-juggling.php#language.types.typecasting]
                \item [link: Conversión a `bool`|http://php.net/manual/es/language.types.boolean.php#language.types.boolean.casting]
                \item [link: Conversión a `int`|http://php.net/manual/es/language.types.integer.php#language.types.integer.casting]
                \item [link: Conversión a `float`|http://php.net/manual/es/language.types.float.php#language.types.float.casting]
                \item [link: Conversión de `string` a número|http://php.net/manual/es/language.types.string.php#language.types.string.conversion]
                \item [link: Conversión a `string`|http://php.net/manual/es/language.types.string.php#language.types.string.casting]
                \item Funciones de obtención de valores
                \begin{longenum}
                    \item [`intval()`](http://php.net/manual/es/function.intval.php)
                    \item [`floatval()`](http://php.net/manual/es/function.floatval.php)
                    \item [`strval()`](http://php.net/manual/es/function.strval.php)
                    \item [`boolval()`](http://php.net/manual/es/function.boolval.php)
                \end{longenum}
                \item Funciones de formateado numérico
                \begin{longenum}
                    \item [`number_format()`](http://php.net/manual/es/function.number-format.php)
                    \item [`money_format()`](http://php.net/manual/es/function.money-format.php)
                    \begin{longenum}
                        \item [`setlocale()`](http://php.net/manual/es/function.setlocale.php)
                    \end{longenum}
                \end{longenum}
            \end{longenum}
            \item Comparaciones
            \begin{longenum}
                \item [Operadores de comparación](http://php.net/manual/es/language.operators.comparison.php)
                \item `==` vs. `===`
                \item [Ternario (`?:`)](http://php.net/manual/es/language.operators.comparison.php#language.operators.comparison.ternary)
                \item [Fusión de `null` (`??`)](https://wiki.php.net/rfc/isset_ternary)
                \item [Reglas de comparación de tipos](http://php.net/manual/es/types.comparisons.php)
            \end{longenum}
        \end{longenum}
    \end{longenum}
    \item **Metodología de la programación**
    \begin{longenum}
        \item Ciclo de vida
        \item Especificación e implementación
        \item Verificación y validación de programas
        \begin{longenum}
            \item Demostración por inducción
        \end{longenum}
        \item Programación funcional
        \begin{longenum}
            \item Especificaciones formales
            \begin{longenum}
                \item Como cálculo
            \end{longenum}
            \item Derivación de programas
            \begin{longenum}
                \item Diseño recursivo
                \begin{longenum}
                    \item Procesos recursivos e iterativos
                    \item Recursividad final
                    \item Técnicas de inmersión
                \end{longenum}
            \end{longenum}
        \end{longenum}
        \item Programación imperativa
        \begin{longenum}
            \item Especificaciones formales
            \begin{longenum}
                \item Como modificación de estados
            \end{longenum}
            \item Derivación de programas
            \begin{longenum}
                \item Diseño iterativo
                \begin{longenum}
                    \item Invariante de un bucle
                    \item Transformación de recursividad final a iterativo
                \end{longenum}
            \end{longenum}
        \end{longenum}
        \item Depuración
        \begin{longenum}
            \item Depuración de programas
            \item El depurador como herramienta de control de errores
        \end{longenum}
    \end{longenum}
    \item **Complejidad algorítmica**
    \begin{longenum}
        \item Introducción
        \item Principio de invarianza
        \item La notación asintótica *O(f(n))*
        \item Órdenes de complejidad
        \item Operaciones entre órdenes de complejidad
        \begin{longenum}
            \item Regla de la suma
            \item Regla del producto
        \end{longenum}
        \item Reglas prácticas para el cálculo de la eficiencia
        \item Resolución de recurrencias
        \begin{longenum}
            \item Reducción de problemas mediante sustracción
            \item Reducción de problemas mediante división
        \end{longenum}
    \end{longenum}
    \item **Programación modular I**
    \begin{longenum}
        \item Introducción
        \begin{longenum}
            \item Descomposición de problemas
        \end{longenum}
        \item Funciones definidas por el usuario
        \begin{longenum}
            \item Definición de funciones con `function`
            \item [link: Paso de argumentos|https://php.net/manual/es/functions.arguments.php]
            \begin{longenum}
                \item Por valor
                \item Por referencia
            \end{longenum}
            \item [Ámbito de variables](http://php.net/language.variables.scope)
            \begin{longenum}
                \item Variables globales
                \begin{longenum}
                    \item Ámbito simple al archivo
                    \item Efectos laterales
                \end{longenum}
                \item Variables locales
            \end{longenum}
            \item Declaraciones de tipos
            \begin{longenum}
                \item [Declaraciones de tipo de argumento](http://php.net/manual/es/functions.arguments.php#functions.arguments.type-declaration)
                \item [Declaraciones de tipo de devolución](http://php.net/manual/es/functions.returning-values.php#functions.returning-values.type-declaration)
                \item [link: Tipos *nullable* (`?`) y `void`|http://php.net/manual/es/migration71.new-features.php]
                \item [link: Tipificación estricta|http://php.net/manual/es/functions.arguments.php#functions.arguments.type-declaration.strict]
            \end{longenum}
        \end{longenum}
        \item Partes de un módulo
        \begin{longenum}
            \item Interfaz
            \item Implementación
            \item Documentación interna
        \end{longenum}
        \item Inclusión de *scripts*
        \begin{longenum}
            \item `require` y `require_once`
            \item `include` e `include_once`
        \end{longenum}
        \item Espacios de nombres
        \item Criterios de descomposición modular
        \begin{longenum}
            \item Abstracción
            \item Ocultación de información
            \item Independencia funcional
            \begin{longenum}
                \item Cohesión
                \item Acoplamiento
            \end{longenum}
            \item Reusabilidad
        \end{longenum}
        \item [link: Diagramas de estructura|https://en.wikipedia.org/wiki/Structure_chart]
    \end{longenum}
    \item **Entrada y salida de información**
    \begin{longenum}
        \item Flujos
        \begin{longenum}
            \item Tipos de flujos
            \begin{longenum}
                \item Flujos de bytes
                \item Flujos de caracteres
            \end{longenum}
            \item Utilización de flujos
        \end{longenum}
        \item La consola
        \begin{longenum}
            \item Entrada desde teclado
            \item Salida a pantalla
        \end{longenum}
        \item Archivos de datos
        \begin{longenum}
            \item Registros
            \item Apertura y cierre de archivos
            \item Modos de acceso
            \item Lectura y escritura de información en archivos
        \end{longenum}
        \item Sistemas de archivos
        \begin{longenum}
            \item Manipulación de los sistemas de archivos
            \item Creación y eliminación de archivos y directorios
        \end{longenum}
    \end{longenum}
    \item **Calidad**
    \begin{longenum}
        \item [Pruebas](https://www.gitbook.com/book/jose/testing-book/)
        \begin{longenum}
            \item Tipos de pruebas
            \begin{longenum}
                \item Unitarias
                \item Funcionales
                \item De aceptación
            \end{longenum}
            \item [link: PHPUnit|https://phpunit.de/]
            \item [link: Cobertura de código|http://codeception.com/docs/11-Codecoverage]
        \end{longenum}
        \item Depuración
        \begin{longenum}
            \item [link: `var_dump()` mejorado|https://github.com/yiisoft/yii2/issues/7352#issuecomment-75024083]
            \item [link: Consola integrada|https://github.com/yiisoft/yii2-shell]
            \item [link: Barra de depuración|https://github.com/yiisoft/yii2-debug]
            \item [link: Depuración con PsySH|https://www.sitepoint.com/interactive-php-debugging-psysh/]
        \end{longenum}
        \item Documentación
        \begin{longenum}
            \item [API documentation generator for Yii2](https://github.com/yiisoft/yii2-apidoc)
            \item [GitHub Pages](https://pages.github.com/)
        \end{longenum}
        \item Mantenimiento y calidad del código
        \begin{longenum}
            \item CodeSniffer
            \item CS_Fixer
            \item [Code Climate](https://codeclimate.com/)
        \end{longenum}
    \end{longenum}
    \item ---
    \item **Estructuras de datos lineales**
    \begin{longenum}
        \item Acceso directo
        \begin{longenum}
            \item Vectores
        \end{longenum}
        \item Acceso secuencial
        \begin{longenum}
            \item Pares
            \item Listas
            \begin{longenum}
                \item Enlazadas
                \item Doblemente enlazadas
            \end{longenum}
            \item Pilas
            \item Colas
        \end{longenum}
    \end{longenum}
    \item **Ordenación y búsqueda**
    \begin{longenum}
        \item Algoritmos de búsqueda
        \begin{longenum}
            \item Búsqueda secuencial
            \item Búsqueda dicotómica
        \end{longenum}
        \item Algoritmos de ordenación
        \begin{longenum}
            \item Inserción directa
            \item Selección directa
            \item Burbuja
            \item *Quicksort*
            \item *Mergesort*
        \end{longenum}
        \item Tablas *Hash*
    \end{longenum}
    \item **Estructuras de datos no lineales**
    \begin{longenum}
        \item Árboles
        \begin{longenum}
            \item Binarios
            \begin{longenum}
                \item Recorridos
                \begin{longenum}
                    \item Preorden
                    \item Inorden
                    \item Postorden
                \end{longenum}
            \end{longenum}
            \item De búsqueda
            \item Montículos
            \begin{longenum}
                \item Algoritmo de ordenación
            \end{longenum}
            \item Generales
            \begin{longenum}
                \item Recorrido en profundidad
                \item Recorrido en anchura
            \end{longenum}
        \end{longenum}
        \item Grafos
        \begin{longenum}
            \item Algoritmo de Dijkstra
            \item Algoritmo de Floyd
        \end{longenum}
    \end{longenum}
    \item **Abstracción de datos**
    \begin{longenum}
        \item Tipos abstractos de datos
        \begin{longenum}
            \item Concepto, terminología y ejemplos
            \item Programación con tipos abstractos de datos
            \begin{longenum}
                \item Modularidad
                \item Refinamientos sucesivos
                \item Programación a gran escala
                \item Programación genérica
            \end{longenum}
        \end{longenum}
        \item Especificación
        \begin{longenum}
            \item Especificaciones algebraicas
            \begin{longenum}
                \item Signatura de un TAD
                \begin{longenum}
                    \item Géneros
                    \item Operaciones
                    \begin{longenum}
                        \item Constructoras
                        \item Selectoras
                    \end{longenum}
                \end{longenum}
                \item Términos
                \item Ecuaciones
            \end{longenum}
            \item Construcción de especificaciones
            \item Verificación con especificaciones algebraicas
        \end{longenum}
        \item Implementación
        \begin{longenum}
            \item Pilas
            \item Colas
            \item Listas
        \end{longenum}
    \end{longenum}
    \item **Programación orientada a objetos**
    \begin{longenum}
        \item Introducción
        \begin{longenum}
            \item Perspectiva histórica
            \item Lenguajes orientados a objetos
        \end{longenum}
        \item Conceptos básicos
        \begin{longenum}
            \item Clase
            \item Objeto
            \begin{longenum}
                \item La antisimetría dato-objeto
            \end{longenum}
            \item Identidad
            \item Estado
            \item Propiedad
            \item Paso de mensajes
            \item Método
            \item Encapsulación
            \item Herencia
            \item Polimorfismo
        \end{longenum}
        \item Uso básico de objetos
        \begin{longenum}
            \item Instanciación
            \begin{longenum}
                \item `new`
                \item `instanceof`
            \end{longenum}
            \item Propiedades
            \begin{longenum}
                \item Acceso y modificación
                \item Tipos de propiedades
                \begin{longenum}
                    \item Predeterminadas
                    \item Dinámicas
                \end{longenum}
            \end{longenum}
            \item [link: Referencias|http://php.net/manual/es/language.references.php]
            \begin{longenum}
                \item Asignación por referencia (`=&`)
            \end{longenum}
            \item [link: Clonación de objetos|http://php.net/manual/es/language.oop5.cloning.php]
            \item [link: Comparación de objetos|http://php.net/manual/es/language.oop5.object-comparison.php]
            \item Destrucción de objetos
            \begin{longenum}
                \item Recolección de basura
            \end{longenum}
            \item Métodos
            \item [link: Constantes|http://php.net/manual/es/language.oop5.constants.php]
            \begin{longenum}
                \item [link: Operador de resolución de ámbito (`::`)|http://php.net/manual/es/language.oop5.paamayim-nekudotayim.php]
            \end{longenum}
            \item [link: *Ejemplo*: La Standard PHP Library (SPL)|https://www.php.net/manual/es/book.spl.php]
        \end{longenum}
        \item Lenguaje UML
        \begin{longenum}
            \item Diagramas de clases
            \item Diagramas de objetos
            \item Diagramas de secuencia
        \end{longenum}
    \end{longenum}
    \item **[link: Diseño de clases|http://php.net/manual/es/language.oop5.php]**
    \begin{longenum}
        \item Encapsulación
        \item [link: Propiedades|http://php.net/manual/es/language.oop5.properties.php]
        \begin{longenum}
            \item Visibilidad
            \begin{longenum}
                \item Pública
                \item Privada
            \end{longenum}
        \end{longenum}
        \item Métodos
        \begin{longenum}
            \item Visibilidad
            \begin{longenum}
                \item Pública
                \item Privada
            \end{longenum}
            \item Referencia `$this`
            \item [link: Constructores y destructores|http://php.net/manual/es/language.oop5.decon.php]
            \item Accesores y mutadores
        \end{longenum}
        \item [link: Constantes|http://php.net/manual/es/language.oop5.constants.php]
        \begin{longenum}
            \item `self`
        \end{longenum}
        \item [link: Miembros estáticos|http://php.net/manual/es/language.oop5.static.php]
        \begin{longenum}
            \item Constantes
            \item Métodos estáticos
            \item Propiedades estáticas
            \item [link: Enlace estático en tiempo de ejecución|http://php.net/manual/es/language.oop5.late-static-bindings.php]
        \end{longenum}
    \end{longenum}
    \item **Composición, herencia y poliformismo**
    \begin{longenum}
        \item Composición de clases
        \item Herencia
        \begin{longenum}
            \item Concepto de herencia
            \item Tipos
            \item Superclases y subclases
            \item Visibilidad protegida
            \item Utilización de clases heredadas
            \item Constructores y herencia
            \item `parent`
        \end{longenum}
        \item Polimorfismo
        \begin{longenum}
            \item El principio de sustitución de Liskov
            \item Sobreescritura de métodos
            \item Sobreescritura de constructores
        \end{longenum}
        \item [link: Herencia vs. composición|https://devexperto.com/herencia-vs-composicion/]
    \end{longenum}
    \item **Programación modular II**
    \begin{longenum}
        \item [link: Las clases como módulos|https://cysingsoft.wordpress.com/2009/06/23/modularidad-cohesion-y-acoplamiento-carlos-fontela/]
        \begin{longenum}
            \item Interfaz de una clase
            \item Métodos *getter* y *setter*
        \end{longenum}
        \item [link: Clases y métodos abstractos|https://www.php.net/manual/es/language.oop5.abstract.php]
        \item [link: Clases y métodos finales|https://www.php.net/manual/es/language.oop5.final.php]
        \item [link: Interfaces|http://php.net/manual/es/language.oop5.interfaces.php]
        \item [link: *Traits*|http://php.net/manual/es/language.oop5.traits.php]
        \item Espacios de nombres y módulos
    \end{longenum}
    \item **Mecanismos de control de errores**
    \begin{longenum}
        \item Control de errores clásico en PHP
        \begin{longenum}
            \item Códigos de error
            \item [link: `set_error_handler()`|https://www.php.net/manual/es/function.set-error-handler.php]
        \end{longenum}
        \item [link: Excepciones|http://php.net/manual/es/language.exceptions.php]
        \begin{longenum}
            \item Errores vs. excepciones
            \item [link: La clase `Exception`|http://php.net/manual/es/class.exception.php]
            \item [link: La clase `Error`|http://php.net/manual/es/class.error.php]
            \item [link: La clase `ErrorException`|http://php.net/manual/es/class.errorexception.php]
            \item Estructura de control `try ... catch`
        \end{longenum}
    \end{longenum}
    \item **Interoperabilidad**
    \begin{longenum}
        \item [link: Versionado semántico|https://semver.org/lang/es/]
        \item [link: Composer|https://getcomposer.org/]
        \begin{longenum}
            \item Paquetes
            \item [Packagist](https://packagist.org/)
            \item [link: Dependencias|https://getcomposer.org/doc/01-basic-usage.md#composer-json-project-setup]
            \begin{longenum}
                \item `composer.json` y `composer.lock`
            \end{longenum}
            \item [link: Versiones y restricciones|https://getcomposer.org/doc/articles/versions.md]
            \begin{longenum}
                \item Versión exacta
                \item Rango (`>`, `>=`, `<`, `<=`, `!=`, ` `, `,`, `||`)
                \item Guión (`-`)
                \item Asterisco (`*`)
                \item Tilde (`~`)
                \item Circunflejo (`^`)
                \item Nombres de rama
                \begin{longenum}
                    \item `dev-master`
                    \item `5.1.x-dev`
                \end{longenum}
                \item [link: Estabilidad mínima|https://getcomposer.org/doc/articles/versions.md#minimum-stability]
                \item [link: Comprobador online de restricciones|https://semver.mwl.be]
            \end{longenum}
            \item Comandos básicos
            \begin{longenum}
                \item [link: `require`|https://getcomposer.org/doc/03-cli.md#require]
                \item [link: `install`|https://getcomposer.org/doc/03-cli.md#install]
                \item [link: `update`|https://getcomposer.org/doc/03-cli.md#update]
            \end{longenum}
            \item Entornos de desarrollo y producción
        \end{longenum}
        \item [Autocarga de clases](http://php.net/manual/es/language.oop5.autoload.php)
        \begin{longenum}
            \item [link: `spl_autoload_register()`|http://php.net/manual/es/function.spl-autoload-register.php]
            \item [link: PSR-4|http://www.php-fig.org/psr/psr-4/]
            \item [link: Autoloader de Composer|https://getcomposer.org/doc/01-basic-usage.md#autoloading]
        \end{longenum}
        \item Ejemplos
        \begin{longenum}
            \item `mpdf/mpdf`
            \item `ramsey/uuid`
            \item `doctrine/inflector`
        \end{longenum}
        \item Recomendaciones PSR del [PHP-FIG](http://www.php-fig.org/) (Framework Interop Group)
        \begin{longenum}
            \item [link: PSR-1: Basic Coding Standard|http://www.php-fig.org/psr/psr-1/]
            \item [link: PSR-2: Coding Style Guide|http://www.php-fig.org/psr/psr-2/]
            \item [link: PSR-4: Autoloading Standard|http://www.php-fig.org/psr/psr-4/]
            \item [link: PSR-5: PHPDoc Standard (borrador)|https://github.com/phpDocumentor/fig-standards/blob/master/proposed/phpdoc.md]
            \item [link: PSR-11: Extended Coding Style Guide (borrador)|https://github.com/php-fig/fig-standards/blob/master/proposed/extended-coding-style-guide.md]
            \item [link: PSR-19: PHPDoc tags (borrador)|https://github.com/php-fig/fig-standards/blob/master/proposed/phpdoc-tags.md]
        \end{longenum}
        \item Ejercicios
        \begin{longenum}
            \item De versionado semántico
            \item De versiones y restricciones
            \item De uso básico de Composer
            \item De buscar paquetes en Packagist que tengan una funcionalidad concreta y usarlos en un ejemplo
        \end{longenum}
    \end{longenum}
    \item **Buenas prácticas en programación orientada a objetos**
    \begin{longenum}
        \item Declaraciones de tipos
        \begin{longenum}
            \item El pseudotipo `void`
            \item Tipificación estricta
            \item Covarianza de retorno y contravarianza de argumentos
        \end{longenum}
        \item Principios de diseño
        \begin{longenum}
            \item Encapsulación y ocultación de información
            \item Diseño orientado a interfaces
            \item Principios *SOLID*
            \begin{longenum}
                \item SRP: Principio de responsabilidad única
                \item OCP: Principio de abierto/cerrado
                \item LSP: Principio de sustitución de Liskov
                \item ISP: Principio de segregación de la interfaz
                \item DIP: Principio de inversión de dependencias
            \end{longenum}
            \item [link: Principio del Menor Conocimiento (o Ley de Demeter)|https://en.wikipedia.org/wiki/Law_of_Demeter]
        \end{longenum}
        \item Patrones de diseño
        \begin{longenum}
            \item De creación
            \item Estructurales
            \item De comportamiento
        \end{longenum}
    \end{longenum}
    \item **Gestión de bases de datos relacionales**
    \begin{longenum}
        \item [link: Componentes de acceso a datos|http://php.net/manual/es/book.pdo.php]
        \begin{longenum}
            \item [link: Clase `PDO`|http://php.net/manual/es/class.pdo.php]
            \begin{longenum}
                \item [link: `__construct(string $dsn [, string $username [, string $password [, array $options ]]])`|http://php.net/manual/es/pdo.construct.php]
                \item [link: `PDOStatement query(string $statement)`|http://php.net/manual/es/pdo.query.php]
                \item [link: `int exec(string $statement)`|http://php.net/manual/es/pdo.exec.php]
                \item [link: `PDOStatement prepare(string $statement [, array $driver_options = array() ])`|http://php.net/manual/es/pdo.prepare.php]
            \end{longenum}
            \item [link: Clase `PDOStatement`|http://php.net/manual/es/class.pdostatement.php]
            \begin{longenum}
                \item [link: `mixed fetch([ int $fetch_style ])`|http://php.net/manual/es/pdostatement.fetch.php]
                \item [link: `mixed fetchAll([ int $fetch_style ])`|http://php.net/manual/es/pdostatement.fetchall.php]
                \item [link: `mixed fetchColumn([ int $column_number = 0 ])`|http://php.net/manual/es/pdostatement.fetchcolumn.php]
                \item [link: `bool execute ([ array $input_parameters ])`|http://php.net/manual/es/pdostatement.execute.php]
                \item [link: `int rowCount(void)`|http://php.net/manual/es/pdostatement.rowcount.php]
            \end{longenum}
            \item Correspondencias de tipos entre SQL y PHP
        \end{longenum}
        \item Establecimiento de conexiones
        \item Recuperación de información
        \begin{longenum}
            \item Ejecución de consultas
            \item Selección de registros
            \item Uso de parámetros
        \end{longenum}
        \item Manipulación de la información
        \begin{longenum}
            \item Altas, bajas y modificaciones
        \end{longenum}
    \end{longenum}
    \item ---
    \item **Introducción al lenguaje de programación Java**
    \begin{longenum}
        \item Introducción
        \item Compilación vs. interpretación
        \begin{longenum}
            \item Máquinas reales vs. virtuales
            \item Código objeto, *bytecode* y archivos `.class`
            \item La plataforma Java
            \begin{longenum}
                \item La máquina virtual de Java (JVM)
                \item La API de Java
            \end{longenum}
            \item El entorno de ejecución de Java (JRE)
            \begin{longenum}
                \item El intérprete `java`
            \end{longenum}
            \item Las herramientas de desarrollo de Java (JDK)
            \begin{longenum}
                \item El compilador `javac`
            \end{longenum}
        \end{longenum}
        \item Características de Java
        \item Tipado estático vs. dinámico
        \item El programa *¡Hola, mundo!* en Java
        \begin{longenum}
            \item El método `main()`
            \item La clase `System`
        \end{longenum}
    \end{longenum}
    \item **Librerías estándar de Java**
    \begin{longenum}
        \item Paquetes
        \item `java.lang`
        \item `java.util`
        \item `java.math`
        \item `java.io`
    \end{longenum}
\end{longenum}
