\begin{center}
\footnotesize
\begin{longtable}[c]{|>{\raggedright}m{4cm}|>{\centering}m{0.7cm}|>{\centering}m{0.7cm}|>{\centering}m{0.7cm}|>{\centering}m{0.7cm}|>{\centering}m{0.7cm}|>{\centering}m{0.7cm}|>{\centering}m{0.7cm}|>{\centering}m{0.7cm}|>{\centering}m{0.7cm}|}
\hline
\textbf{Unidades didácticas} & \ra1 & \ra2 & \ra3 & \ra4 & \ra5 & \ra6 & \ra7 & \ra8 & \ra9\tabularnewline
\hline
\hline
\endhead
1. Introducción & $\times$ &  &  &  &  &  &  &  &  \tabularnewline
\hline
2. Programación funcional I & \ce{1a}\ \ce{1b}\ \ce{1c}\ \ce{1e}\ \ce{1f}\ \ce{1g}\ \ce{1i}\ &  & \ce{3f}\ \ce{3g}\ &  &  & $\times$ &  &  &  \tabularnewline
\hline
3. Programación funcional II & \ce{1a}\ \ce{1b}\ \ce{1c}\ &  & \ce{3f}\ \ce{3g}\ &  &  & $\times$ &  &  &  \tabularnewline
\hline
4. Programación imperativa & \ce{1a}\ \ce{1b}\ \ce{1c}\ &  & \ce{3f}\ \ce{3g}\ &  &  & $\times$ &  &  &  \tabularnewline
\hline
5. Programación estructurada & \ce{1a}\ \ce{1b}\ \ce{1c}\ &  & \ce{3a}\ \ce{3f}\ \ce{3g}\ &  &  & $\times$ &  &  &  \tabularnewline
\hline
6. Tipos de datos & \ce{1d}\ \ce{1h}\ &  & \ce{3f}\ \ce{3g}\ &  &  & \ce{6g}\ &  &  &  \tabularnewline
\hline
7. Metodología de la programación & \ce{1a}\ \ce{1b}\ \ce{1c}\ &  & \ce{3f}\ \ce{3g}\ &  &  & $\times$ &  &  &  \tabularnewline
\hline
8. Programación modular I & \ce{1a}\ \ce{1b}\ \ce{1c}\ &  & \ce{3f}\ \ce{3g}\ &  &  & $\times$ &  &  &  \tabularnewline
\hline
9. Entrada y salida de información & \ce{1a}\ \ce{1b}\ \ce{1c}\ &  & \ce{3f}\ \ce{3g}\ &  & \ce{5a}\ \ce{5b}\ \ce{5c}\ \ce{5d}\ \ce{5e}\ & \ce{6h}\ \ce{6i}\ &  &  &  \tabularnewline
\hline
10. Calidad I & \ce{1a}\ \ce{1b}\ \ce{1c}\ &  & \ce{3f}\ \ce{3g}\ &  &  &  &  &  &  \tabularnewline
\hline
11. Complejidad algorítmica &  &  &  &  &  &  &  &  &  \tabularnewline
\hline
12. Introducción al lenguaje Java & \ce{1a}\ \ce{1b}\ \ce{1c}\ &  & \ce{3f}\ \ce{3g}\ &  &  &  &  &  &  \tabularnewline
\hline
13. Programación orientada a objetos & \ce{1a}\ \ce{1b}\ \ce{1c}\ & \ce{2a}\ \ce{2b}\ \ce{2c}\ \ce{2d}\ \ce{2f}\ \ce{2h}\ \ce{2i}\ & \ce{3f}\ \ce{3g}\ &  &  & \ce{6a}\ &  &  &  \tabularnewline
\hline
14. Diseño de clases & \ce{1a}\ \ce{1b}\ \ce{1c}\ & \ce{2e}\ & \ce{3f}\ \ce{3g}\ & \ce{4a}\ \ce{4b}\ \ce{4c}\ \ce{4d}\ \ce{4e}\ \ce{4f}\ \ce{4h}\ &  &  &  &  &  \tabularnewline
\hline
15. Composición, herencia y poliformismo & \ce{1a}\ \ce{1b}\ \ce{1c}\ &  & \ce{3f}\ \ce{3g}\ & \ce{4g}\ &  &  & \ce{7a}\ \ce{7b}\ \ce{7c}\ \ce{7d}\ \ce{7e}\ \ce{7f}\ \ce{7g}\ \ce{7h}\ &  &  \tabularnewline
\hline
16. Abstracción de datos & \ce{1a}\ \ce{1b}\ \ce{1c}\ &  & \ce{3f}\ \ce{3g}\ &  &  & $\times$ &  &  &  \tabularnewline
\hline
17. Programación modular II & \ce{1a}\ \ce{1b}\ \ce{1c}\ &  & \ce{3f}\ \ce{3g}\ & \ce{4i}\ \ce{4j}\ &  &  &  &  &  \tabularnewline
\hline
18. Programación genérica & \ce{1a}\ \ce{1b}\ \ce{1c}\ &  & \ce{3f}\ \ce{3g}\ &  &  & \ce{6f}\ &  &  &  \tabularnewline
\hline
19. Estructuras de datos lineales & \ce{1a}\ \ce{1b}\ \ce{1c}\ &  & \ce{3f}\ \ce{3g}\ &  &  & \ce{6a}\ &  &  &  \tabularnewline
\hline
20. Ordenación y búsqueda & \ce{1a}\ \ce{1b}\ \ce{1c}\ &  & \ce{3f}\ \ce{3g}\ &  &  & $\times$ &  &  &  \tabularnewline
\hline
21. Estructuras de datos no lineales & \ce{1a}\ \ce{1b}\ \ce{1c}\ &  & \ce{3f}\ \ce{3g}\ &  &  & $\times$ &  &  &  \tabularnewline
\hline
22. Control de excepciones & \ce{1a}\ \ce{1b}\ \ce{1c}\ &  & \ce{3d}\ \ce{3f}\ \ce{3g}\ &  &  &  &  &  &  \tabularnewline
\hline
23. Java Collections Framework & \ce{1a}\ \ce{1b}\ \ce{1c}\ &  & \ce{3f}\ \ce{3g}\ &  &  & \ce{6a}\ \ce{6b}\ \ce{6c}\ \ce{6d}\ \ce{6e}\ \ce{6f}\ &  &  &  \tabularnewline
\hline
24. Principios y patrones de diseño & \ce{1a}\ \ce{1b}\ \ce{1c}\ &  & \ce{3f}\ \ce{3g}\ &  &  &  &  &  &  \tabularnewline
\hline
25. Calidad II & \ce{1a}\ \ce{1b}\ \ce{1c}\ &  & \ce{3f}\ \ce{3g}\ &  &  &  &  &  &  \tabularnewline
\hline
26. Gestión de bases de datos relacionales & \ce{1a}\ \ce{1b}\ \ce{1c}\ &  & \ce{3f}\ \ce{3g}\ &  &  & $\times$ &  & \ce{8a}\ \ce{8b}\ \ce{8c}\ \ce{8d}\ \ce{8e}\ \ce{8f}\ \ce{8g}\ \ce{8h}\ & \ce{9a}\ \ce{9b}\ \ce{9c}\ \ce{9d}\ \ce{9e}\ \ce{9f}\ \ce{9g}\ \tabularnewline
\hline
27. Programación de interfaces gráficas de usuario & \ce{1a}\ \ce{1b}\ \ce{1c}\ &  & \ce{3f}\ \ce{3g}\ &  & \ce{5f}\ \ce{5g}\ \ce{5h}\ & $\times$ &  &  &  \tabularnewline
\hline
\end{longtable}
\par\end{center}
