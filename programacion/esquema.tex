\begin{longenum}
    \item \textbf{\textsc{Introducción}} \ev1\ \ra1\ (est: 2021\==09\==20)
    \begin{longenum}
        \item Conceptos básicos
        \begin{longenum}
            \item Informática
            \begin{longenum}
                \item Procesamiento automático
            \end{longenum}
            \item Ordenador
            \begin{longenum}
                \item Definición
                \item Funcionamiento básico
                \begin{longenum}
                    \item Elementos funcionales
                    \item Ciclo de instrucción
                    \item Representación de información
                    \begin{longenum}
                        \item Codificación interna
                        \begin{longenum}
                            \item Sistema binario
                        \end{longenum}
                        \item Codificación externa
                        \begin{longenum}
                            \item ASCII
                            \item Unicode
                        \end{longenum}
                    \end{longenum}
                \end{longenum}
            \end{longenum}
            \item Problema
            \begin{longenum}
                \item Generalización
                \item Ejemplares de un problema
                \item Dominio de definición
                \item Jerarquías de generalización
            \end{longenum}
            \item Algoritmo
            \begin{longenum}
                \item Definición
                \item Características
                \item Representación
                \begin{longenum}
                    \item Ordinograma
                    \item Pseudocódigo
                \end{longenum}
                \item Cualidades deseables
                \item Computabilidad
                \item Corrección
                \item Complejidad
            \end{longenum}
            \item Programa
            \item Lenguaje de programación
        \end{longenum}
        \item Paradigmas de programación
        \begin{longenum}
            \item Definición
            \item Imperativo
            \begin{longenum}
                \item Estructurado
                \item Orientado a objetos
            \end{longenum}
            \item Declarativo
            \begin{longenum}
                \item Funcional
                \item Lógico
                \item De bases de datos
            \end{longenum}
        \end{longenum}
        \item Lenguajes de programación
        \begin{longenum}
            \item Definición
            \begin{longenum}
                \item Sintaxis
                \begin{longenum}
                    \item Notación EBNF
                \end{longenum}
                \item Semántica estática
                \item Semántica dinámica
            \end{longenum}
            \item Evolución histórica 
            \item Clasificación
            \begin{longenum}
                \item Por nivel
                \item Por generación
                \item Por propósito
                \item Por paradigma
            \end{longenum}
        \end{longenum}
        \item Traductores
        \begin{longenum}
            \item Definición
            \item Compiladores
            \begin{longenum}
                \item Ensambladores
            \end{longenum}
            \item Intérpretes
            \begin{longenum}
                \item Interactivos (\textit{REPL})
            \end{longenum}
        \end{longenum}
        \item Resolución de problemas mediante programación
        \begin{longenum}
            \item Especificación
            \item Análisis del problema
            \item Diseño del algoritmo
            \item Verificación
            \item Estudio de la eficiencia
            \item Codificación
            \item Traducción y ejecución
            \item Pruebas
            \item Depuración
            \item Documentación
            \item Mantenimiento
            \item Ingeniería del software
        \end{longenum}
        \item Entornos integrados de desarrollo
        \begin{longenum}
            \item Definición
            \item Editores de textos
            \item Editores vs. IDE
            \item Visual Studio Code
        \end{longenum}
    \end{longenum}
    \item \textbf{\textsc{Expresiones}} \ce{1a}\ \ce{1b}\ \ce{1c}\ \ce{1e}\ \ce{1f}\ \ce{1g}\ \ce{1i}\ \ce{3f}\ \ce{3g}\ \ev1\ \ra1\ \ra3\ \ra6\ (est: 2021\==09\==27)
    \begin{longenum}
        \item El lenguaje de programación Python
        \begin{longenum}
            \item Historia
            \item Características principales
            \item Instalación
            \item Funcionamiento del intérprete
            \begin{longenum}
                \item Entrada y salida del intérprete
            \end{longenum}
            \item Instalación de Visual Studio Code
            \begin{longenum}
                \item Configuración básica de Visual Studio Code
            \end{longenum}
        \end{longenum}
        \item Elementos de un programa
        \begin{longenum}
            \item Expresiones y sentencias
            \item Sintaxis y semántica de las expresiones
        \end{longenum}
        \item Valores
        \begin{longenum}
            \item Datos, tipos y valores
            \item Evaluación de expresiones
            \item Expresión canónica y forma normal
            \item Formas normales y evaluación
            \item Literales
            \item Identificadores
        \end{longenum}
        \item Operaciones
        \begin{longenum}
            \item Clasificación
            \item Operadores
            \begin{longenum}
                \item Aridad de operadores
                \item Notación de los operadores
                \item Paréntesis
                \item Prioridad de operadores
                \item Asociatividad de operadores
                \item Paréntesis y orden de evaluación
                \item Tipos de operandos
            \end{longenum}
            \item Funciones
            \begin{longenum}
                \item Funciones con varios parámetros
                \item Evaluación de expresiones con funciones
                \item Composición de operaciones y funciones
            \end{longenum}
            \item Métodos
        \end{longenum}
        \item Otros conceptos sobre operaciones
        \begin{longenum}
            \item Tipos polimórficos y operaciones polimórficas
            \item Sobrecarga de operaciones
            \item Equivalencia entre formas de operaciones
            \item Igualdad de operaciones
        \end{longenum}
        \item Operaciones predefinidas
        \begin{longenum}
            \item Operadores predefinidos
            \begin{longenum}
                \item Operadores aritméticos
                \item Operadores de cadenas
            \end{longenum}
            \item Funciones predefinidas
            \begin{longenum}
                \item Funciones matemáticas y módulos
                \begin{longenum}
                    \item El módulo \texttt{operator}
                \end{longenum}
            \end{longenum}
            \item Métodos predefinidos
        \end{longenum}
    \end{longenum}
    \item \textbf{\textsc{Programación funcional (I)}} \ce{1a}\ \ce{1b}\ \ce{1c}\ \ce{1e}\ \ce{1f}\ \ce{1g}\ \ce{1i}\ \ce{3f}\ \ce{3g}\ \ev1\ \ra1\ \ra3\ \ra6\ (est: 2021\==10\==04)
    \begin{longenum}
        \item Introducción
        \begin{longenum}
            \item Concepto
            \item Transparencia referencial
            \item Modelo de ejecución
            \begin{longenum}
                \item Modelo de sustitución
            \end{longenum}
        \end{longenum}
        \item Tipos de datos
        \begin{longenum}
            \item Concepto
            \item Tipo de un dato
            \item \texttt{type}
            \item Sistemas de tipos
            \begin{longenum}
                \item Errores de tipos
                \item Tipado fuerte vs. débil
            \end{longenum}
            \item Tipos de datos básicos
            \begin{longenum}
                \item Números
                \item Cadenas
                \item Funciones
            \end{longenum}
            \item Conversión de tipos
        \end{longenum}
        \item Álgebra de Boole
        \begin{longenum}
            \item El tipo de dato \textit{booleano}
            \item Operadores relacionales
            \item Operadores lógicos
            \begin{longenum}
                \item Tablas de verdad
            \end{longenum}
            \item Axiomas
            \begin{longenum}
                \item Traducción a Python
            \end{longenum}
            \item Teoremas fundamentales
            \begin{longenum}
                \item Traducción a Python
            \end{longenum}
            \item El operador ternario
        \end{longenum}
        \item Definiciones
        \begin{longenum}
            \item Introducción
            \item Identificadores y ligaduras (\textit{binding})
            \begin{longenum}
                \item Reglas léxicas
                \item Tipo de un identificador
                \item Las funciones como datos
            \end{longenum}
            \item Espacios de nombres
            \item Marcos (\textit{frames})
            \item Evaluación de expresiones con identificadores
            \begin{longenum}
                \item Resolución de identificadores
            \end{longenum}
            \item \textit{Scripts}
        \end{longenum}
        \item Documentación interna
        \begin{longenum}
            \item Identificadores significativos
            \item Comentarios
        \end{longenum}
    \end{longenum}
    \item \textbf{\textsc{Abstracciones funcionales}} \ce{1a}\ \ce{1b}\ \ce{1c}\ \ce{1e}\ \ce{1f}\ \ce{1g}\ \ce{1i}\ \ce{3f}\ \ce{3g}\ \ev1\ \ra1\ \ra3\ \ra6\ (est: 2021\==10\==11)
    \begin{longenum}
        \item Abstracciones lambda
        \begin{longenum}
            \item Expresiones lambda
            \item Parámetros y cuerpos
            \item Aplicación funcional
            \begin{longenum}
                \item Evaluación de una aplicación funcional
                \item Llamadas a funciones
            \end{longenum}
            \item Variables ligadas y libres
        \end{longenum}
        \item Ámbitos léxicos
        \begin{longenum}
            \item Ámbitos
            \item Ámbito de creación de una ligadura
            \item Ámbito de una ligadura
            \item Ámbitos y espacios de nombres
            \item Ámbito de un identificador
            \item Ámbito de un parámetro
            \item Ámbito de una variable ligada
        \end{longenum}
        \item Evaluación
        \begin{longenum}
            \item Entorno (\textit{environment})
            \begin{longenum}
                \item Ámbitos, marcos y entornos
            \end{longenum}
            \item Evaluación de expresiones con entornos
            \item Evaluación de expresiones lambda con entornos
            \begin{longenum}
                \item Variables \textit{sombreadas}
                \item Renombrado de parámetros
                \item Visualización en \textit{Pythontutor}
            \end{longenum}
            \item Estrategias de evaluación
            \begin{longenum}
                \item Orden de evaluación
                \begin{longenum}
                    \item Orden aplicativo
                    \item Orden normal
                \end{longenum}
                \item Composición de funciones
                \item Evaluación estricta y no estricta
            \end{longenum}
        \end{longenum}
        \item Abstracciones funcionales
        \begin{longenum}
            \item Pureza
            \item Las funciones como abstracciones
            \begin{longenum}
                \item Especificaciones de funciones
            \end{longenum}
        \end{longenum}
    \end{longenum}
    \item \textbf{\textsc{Programación funcional (II)}} \ce{1a}\ \ce{1b}\ \ce{1c}\ \ce{3f}\ \ce{3g}\ \ev1\ \ra1\ \ra3\ \ra6\ (est: 2021\==10\==18)
    \begin{longenum}
        \item Computabilidad
        \begin{longenum}
            \item Funciones y procesos
            \item Funciones \textit{ad-hoc}
            \item Funciones recursivas
            \begin{longenum}
                \item Definición
                \item Casos base y casos recursivos
                \item El factorial
                \item Diseño de funciones recursivas
                \begin{longenum}
                    \item Pensamiento optimista
                    \item Descomposición del problema
                    \item Identificación de problemas no reducibles
                \end{longenum}
                \item Recursividad lineal
                \begin{longenum}
                    \item Procesos recursivos lineales
                    \item Procesos iterativos linales
                \end{longenum}
                \item Recursividad múltiple
                \item Recursividad final y no final
            \end{longenum}
            \item La pila de control
            \item Un lenguaje Turing-completo
        \end{longenum}
        \item Tipos de datos recursivos
        \begin{longenum}
            \item Concepto
            \item Cadenas
            \item Listas
            \item Rangos
            \item Conversión a lista
        \end{longenum}
        \item Funciones de orden superior
        \begin{longenum}
            \item Concepto
            \item \texttt{map}
            \item \texttt{filter}
            \item \texttt{reduce}
            \item Expresiones generadoras
        \end{longenum}
    \end{longenum}
    \item \textbf{\textsc{Programación imperativa}} \ce{1a}\ \ce{1b}\ \ce{1c}\ \ce{3f}\ \ce{3g}\ \ce{5a}\ \ce{5b}\ \ce{5c}\ \ce{5d}\ \ce{5e}\ \ce{6h}\ \ce{6i}\ \ev1\ \ra1\ \ra3\ \ra5\ \ra6\ (est: 2021\==10\==25)
    \begin{longenum}
        \item Modelo de ejecución
        \begin{longenum}
            \item Máquina de estados
            \item Sentencias
            \item Secuencia de sentencias
        \end{longenum}
        \item Asignación destructiva
        \begin{longenum}
            \item Identidad
            \begin{longenum}
                \item \texttt{id}
            \end{longenum}
            \item Variables y referencias
            \item Estado
            \item Marcos en programación imperativa
            \item Sentencia de asignación
            \item Evaluación de expresiones con variables
            \item Constantes
            \item Tipado estático vs. dinámico
            \item Asignación compuesta
            \item Asignación múltiple
        \end{longenum}
        \item Mutabilidad
        \begin{longenum}
            \item Estado de un dato
            \item Tipos mutables e inmutables
            \begin{longenum}
                \item Inmutables
                \item Mutables
            \end{longenum}
            \item Alias de variables
            \begin{longenum}
                \item Recolección de basura
                \item \texttt{is}
            \end{longenum}
        \end{longenum}
        \item Cambios de estado ocultos
        \begin{longenum}
            \item Funciones puras
            \item Funciones impuras
            \item Efectos laterales
            \item Transparencia referencial
            \item Entrada y salida por consola
            \begin{longenum}
                \item \texttt{print}
                \begin{longenum}
                    \item Paso de argumentos por palabras clave
                    \item El valor \texttt{None}
                \end{longenum}
                \item \texttt{input}
            \end{longenum}
            \item Ejecución de \textit{scripts} por lotes
            \begin{longenum}
                \item Argumentos de la línea de órdenes
            \end{longenum}
            \item Entrada y salida por archivos 
            \begin{longenum}
                \item \texttt{open}
                \item \texttt{read}
                \item \texttt{readline}
                \item \texttt{write}
                \item \texttt{seek} y \texttt{tell}
                \item \texttt{close}
            \end{longenum}
        \end{longenum}
        \item Saltos
        \begin{longenum}
            \item Incondicionales
            \item Condicionales
        \end{longenum}
    \end{longenum}
    \item \textbf{\textsc{Programación estructurada}} \ce{1a}\ \ce{1b}\ \ce{1c}\ \ce{3a}\ \ce{3f}\ \ce{3g}\ \ev1\ \ra1\ \ra3\ \ra6\ (est: 2021\==11\==01)
    \begin{longenum}
        \item Aspectos teóricos de la programación estructurada
        \begin{longenum}
            \item Programación estructurada
            \item Programa restringido
            \item Programa propio
            \item Estructura
            \item Programa estructurado
            \begin{longenum}
                \item Ventajas de los programas estructurados
            \end{longenum}
            \item Teorema de Böhm-Jacopini
        \end{longenum}
        \item Estructuras básicas de control en Python
        \begin{longenum}
            \item Secuencia
            \item Selección
            \item Iteración
            \item Otras sentencias de control
            \begin{longenum}
                \item \texttt{break}
                \item \texttt{continue}
                \item Excepciones
                \begin{longenum}
                    \item Gestión de excepciones
                \end{longenum}
                \item Gestores de contexto 
            \end{longenum}
        \end{longenum}
        \item Metodología de la programación estructurada
        \begin{longenum}
            \item Recursos abstractos
            \item Diseño descendente
            \item Refinamiento sucesivo
        \end{longenum}
        \item Funciones imperativas
        \begin{longenum}
            \item Definición de funciones imperativas
            \item Llamadas a funciones imperativas
            \item Paso de argumentos
            \item La sentencia \texttt{return}
            \item Ámbito de variables
            \begin{longenum}
                \item Variables locales
                \item Variables globales
                \begin{longenum}
                    \item \texttt{global}
                    \item Efectos laterales
                \end{longenum}
            \end{longenum}
            \item Funciones locales a funciones
            \begin{longenum}
                \item \texttt{nonlocal}
            \end{longenum}
        \end{longenum}
    \end{longenum}
    \item \textbf{\textsc{Tipos de datos estructurados}} \ce{1d}\ \ce{1h}\ \ce{3f}\ \ce{3g}\ \ce{5b}\ \ce{6g}\ \ev1\ \ra1\ \ra3\ \ra6\ (est: 2021\==11\==08)
    \begin{longenum}
        \item Conceptos básicos
        \begin{longenum}
            \item Introducción
            \item \textit{Hashables}
            \item Iterables
            \item Iteradores
            \begin{longenum}
                \item Expresiones generadoras
                \item El bucle \texttt{for}
                \item El módulo \texttt{itertools}
            \end{longenum}
        \end{longenum}
        \item Secuencias
        \begin{longenum}
            \item Concepto de secuencia
            \item Operaciones comunes
            \item Inmutables
            \begin{longenum}
                \item Cadenas (\texttt{str})
                \begin{longenum}
                    \item Formateado de cadenas
                    \item Expresiones regulares 
                \end{longenum}
                \item Tuplas
                \item Rangos
            \end{longenum}
            \item Mutables
            \begin{longenum}
                \item Listas
                \begin{longenum}
                    \item Listas por compresión
                \end{longenum}
                \item Operaciones mutadoras
            \end{longenum}
        \end{longenum}
        \item Estructuras no secuenciales
        \begin{longenum}
            \item Conjuntos (\texttt{set} y \texttt{frozenset})
            \begin{longenum}
                \item Operaciones
                \item Operaciones sobre conjuntos mutables
            \end{longenum}
            \item Diccionarios (\texttt{dict})
            \begin{longenum}
                \item Operaciones
            \end{longenum}
        \end{longenum}
    \end{longenum}
    \item \textbf{\textsc{Programación modular (I)}} \ce{1a}\ \ce{1b}\ \ce{1c}\ \ce{3f}\ \ce{3g}\ \ev1\ \ra1\ \ra3\ \ra6\ (est: 2021\==11\==15)
    \begin{longenum}
        \item Introducción
        \begin{longenum}
            \item Modularidad
            \item Descomposición de problemas
            \item Beneficios de la modularidad
        \end{longenum}
        \item Diseño modular
        \begin{longenum}
            \item Creadores y usuarios
            \item Partes de un módulo
            \begin{longenum}
                \item Interfaz
                \item Implementación
            \end{longenum}
            \item Diagramas de estructura
        \end{longenum}
        \item Programación modular en Python
        \begin{longenum}
            \item \textit{Scripts} como módulos
            \item Importación de módulos
            \item Módulos como \textit{scripts}
            \item La librería estándar
            \item Paquetes \opcional\
            \item Documentación interna \opcional\
        \end{longenum}
        \item Criterios de descomposición modular
        \begin{longenum}
            \item Abstracción
            \item Ocultación de información
            \item Independencia funcional
            \begin{longenum}
                \item Cohesión
                \item Acoplamiento
            \end{longenum}
            \item Reusabilidad
        \end{longenum}
    \end{longenum}
    \item \textbf{\textsc{Abstracción de datos}} \ce{1a}\ \ce{1b}\ \ce{1c}\ \ce{3f}\ \ce{3g}\ \ev1\ \ra1\ \ra3\ \ra6\ (est: 2021\==11\==22)
    \begin{longenum}
        \item Introducción
        \begin{longenum}
            \item Introducción
            \item Tipos abstractos de datos
        \end{longenum}
        \item Especificaciones
        \begin{longenum}
            \item Sintaxis
            \item Operaciones
            \item Ejemplos
        \end{longenum}
        \item Implementaciones
        \begin{longenum}
            \item Implementaciones
        \end{longenum}
        \item Niveles y barreras de abstracción
        \begin{longenum}
            \item Niveles de abstracción
            \item Barreras de abstracción
            \item Propiedades de los datos
        \end{longenum}
        \item Las funciones como datos
        \begin{longenum}
            \item Clausuras
            \item Representación funcional
            \item Estado interno
            \item Paso de mensajes
            \item Especificación de datos abstractos con estado interno
        \end{longenum}
        \item Abstracción de datos y modularidad
        \begin{longenum}
            \item El tipo abstracto como módulo
        \end{longenum}
    \end{longenum}
    \item \textbf{\textsc{Calidad (I)}} \ce{1a}\ \ce{1b}\ \ce{1c}\ \ce{3f}\ \ce{3g}\ \ev1\ \opcional\ \ra1\ \ra3\ (est: 2021\==11\==29)
    \begin{longenum}
        \item Documentación interna
        \begin{longenum}
            \item Concepto
            \item Comentarios
            \item \textit{Docstrings}
            \item \texttt{pydoc}
            \item Estándares de codificación
            \begin{longenum}
                \item PEP 8
                \item \texttt{pylint}
                \item \texttt{autopep8}
            \end{longenum}
        \end{longenum}
        \item Depuración
        \begin{longenum}
            \item \texttt{print}
            \item Depuración en el IDE
        \end{longenum}
        \item Pruebas
        \begin{longenum}
            \item Enfoques de pruebas
            \begin{longenum}
                \item Pruebas de caja blanca
                \item Pruebas de caja negra
            \end{longenum}
            \item Estrategias de pruebas
            \begin{longenum}
                \item Unitarias
                \item Funcionales
                \item De aceptación
            \end{longenum}
            \item \texttt{doctest}
            \item \texttt{pytest}
            \item Desarrollo conducido por pruebas
            \begin{longenum}
                \item Ciclo de desarrollo
                \item Ventajas
            \end{longenum}
        \end{longenum}
    \end{longenum}
    \item \textbf{\textsc{Programación orientada a objetos}} \ce{1a}\ \ce{1b}\ \ce{1c}\ \ce{2a}\ \ce{2b}\ \ce{2c}\ \ce{2d}\ \ce{2f}\ \ce{2h}\ \ce{2i}\ \ce{3f}\ \ce{3g}\ \ce{6a}\ \ev2\ \ra1\ \ra2\ \ra3\ \ra6\ (est: 2022\==01\==10)
    \begin{longenum}
        \item Introducción
        \begin{longenum}
            \item Recapitulación
            \item La metáfora del objeto
        \end{longenum}
        \item Clases y objetos
        \begin{longenum}
            \item Clases
            \item Objetos
            \item Estado
            \begin{longenum}
                \item Atributos
            \end{longenum}
            \item La antisimetría dato-objeto
        \end{longenum}
        \item Paso de mensajes
        \begin{longenum}
            \item Introducción
            \item Ejecución de métodos
            \item Definición de métodos
            \item Métodos \_mágicos\_ y constructores
        \end{longenum}
        \item Identidad e igualdad
        \begin{longenum}
            \item Identidad
            \item Igualdad
            \begin{longenum}
                \item \texttt{\_\_eq\_\_}
                \item \texttt{\_\_hash\_\_}
            \end{longenum}
            \item Otros métodos mágicos
            \begin{longenum}
                \item \texttt{\_\_repr\_\_}
                \item \texttt{\_\_str\_\_}
            \end{longenum}
        \end{longenum}
        \item Encapsulación
        \begin{longenum}
            \item La encapsulación como mecanismo de agrupamiento
            \item La encapsulación como mecanismo de protección de datos
            \begin{longenum}
                \item Visibilidad
                \item Accesores y mutadores
                \item Invariantes de clase
                \item Interfaz y especificación de una clase
                \item Asertos
            \end{longenum}
        \end{longenum}
        \item Miembros de clase
        \begin{longenum}
            \item Variables de clase
            \item Métodos estáticos
        \end{longenum}
    \end{longenum}
    \item \textbf{\textsc{Relaciones entre clases}} \ce{1a}\ \ce{1b}\ \ce{1c}\ \ce{3f}\ \ce{3g}\ \ce{4g}\ \ce{7a}\ \ce{7b}\ \ce{7c}\ \ce{7d}\ \ce{7e}\ \ce{7f}\ \ce{7g}\ \ce{7h}\ \ev2\ \ra1\ \ra3\ \ra4\ \ra7\ (est: 2022\==01\==17)
    \begin{longenum}
        \item Relaciones básicas
        \begin{longenum}
            \item Introducción
            \item Asociación
            \item Dependencia
            \item Agregación
            \item Composición
        \end{longenum}
        \item Herencia
        \begin{longenum}
            \item Generalización
            \item Modos
            \begin{longenum}
                \item Herencia simple
                \item Visibilidad de miembros y herencia
                \begin{longenum}
                    \item Visibilidad protegida
                \end{longenum}
                \item La clase \texttt{object}
                \item Herencia múltiple
            \end{longenum}
        \end{longenum}
        \item Polimorfismo
        \begin{longenum}
            \item Concepto
            \item Principio de sustitución de Liskov
            \item \textit{Duck typing}
            \item Sobreescritura de métodos
            \begin{longenum}
                \item Polimorfismo y métodos redefinidos
            \end{longenum}
            \item Ligadura dinámica
            \item \texttt{super}
            \item Sobreescritura de constructores
            \item Clases abstractas y métodos abstractos
        \end{longenum}
        \item Herencia vs. composición
    \end{longenum}
    \item \textbf{\textsc{Introducción a la tecnología Java}} \ce{1a}\ \ce{1b}\ \ce{1c}\ \ce{1e}\ \ce{1f}\ \ce{2b}\ \ce{2i}\ \ce{3f}\ \ce{3g}\ \ev2\ \ra1\ \ra2\ (est: 2022\==01\==24)
    \begin{longenum}
        \item Introducción
        \begin{longenum}
            \item Historia
            \item Versiones
            \item Características principales
        \end{longenum}
        \item La tecnología Java
        \begin{longenum}
            \item Máquinas reales vs. virtuales
            \item Código objeto (\textit{bytecode})
            \item La plataforma Java
            \begin{longenum}
                \item La máquina virtual de Java (JVM)
                \item La API de Java
            \end{longenum}
            \item Las herramientas de desarrollo de Java (JDK)
            \begin{longenum}
                \item El compilador \texttt{javac}
                \item El intérprete interactivo \texttt{jshell}
            \end{longenum}
            \item El entorno de ejecución de Java (JRE)
            \begin{longenum}
                \item El intérprete \texttt{java}
            \end{longenum}
        \end{longenum}
        \item El primer programa Java
        \begin{longenum}
            \item Tipado estático vs. dinámico
            \item El método \texttt{main}
            \item La clase principal
            \item La clase \texttt{System}
            \item El paquete \texttt{java.lang}
            \item El objeto \texttt{out}
            \item El método \texttt{println}
        \end{longenum}
    \end{longenum}
    \item \textbf{\textsc{Elementos básicos del lenguaje Java}} \ce{1a}\ \ce{1b}\ \ce{1c}\ \ce{1d}\ \ce{1e}\ \ce{1f}\ \ce{1h}\ \ce{1i}\ \ce{2b}\ \ce{2i}\ \ce{3a}\ \ce{3b}\ \ce{3c}\ \ce{3e}\ \ce{3f}\ \ce{3g}\ \ev2\ \ra1\ \ra2\ \ra3\ \ra5\ (est: 2022\==01\==31)
    \begin{longenum}
        \item Tipos y valores en Java
        \begin{longenum}
            \item Introducción
            \item Tipos primitivos
            \begin{longenum}
                \item Booleanos
                \item Integrales
                \begin{longenum}
                    \item Operadores de integrales
                \end{longenum}
                \item De coma flotante
                \begin{longenum}
                    \item Operadores de coma flotante
                \end{longenum}
                \item Subtipado
                \begin{longenum}
                    \item Subtipado entre tipos primitivos
                    \item Subtipado entre tipos referencia
                \end{longenum}
                \item Conversiones entre datos primitivos
                \begin{longenum}
                    \item \textit{Casting}
                    \item De ampliación (\textit{widening})
                    \item De restricción (\textit{narrowing})
                \end{longenum}
                \item Promociones numéricas
            \end{longenum}
            \item Tipos referencia
            \begin{longenum}
                \item Nulo
                \item Acceso a miembros
                \begin{longenum}
                    \item Llamadas a métodos sobrecargados
                \end{longenum}
            \end{longenum}
        \end{longenum}
        \item Variables en Java
        \begin{longenum}
            \item Introducción
            \item Variables de tipos primitivos
            \item Variables de tipos referencia
            \begin{longenum}
                \item Tipo estático y tipo dinámico
            \end{longenum}
            \item Declaración de variables
            \begin{longenum}
                \item Inicialización y asignación de variables
                \begin{longenum}
                    \item Declaración vs. definición
                    \item Operadores de asignación compuesta
                    \item Operadores de incremento y decremento
                    \item Inicialización y asignación con literales numéricos
                \end{longenum}
                \item Inferencia de tipos
                \item Constantes
                \item Declaración de variables de tipo referencia
            \end{longenum}
        \end{longenum}
        \item Estructuras de control
        \begin{longenum}
            \item Bloques
            \item \texttt{if}
            \item \texttt{switch}
            \item \texttt{while}
            \item \texttt{for}
            \item \texttt{do ... while}
            \item \texttt{break} y \texttt{continue}
        \end{longenum}
        \item Entrada/salida
        \begin{longenum}
            \item Flujos \texttt{System.in}, \texttt{System.out} y \texttt{System.err}
            \item \texttt{java.util.Scanner}
        \end{longenum}
    \end{longenum}
    \item \textbf{\textsc{Programación orientada a objetos en Java}} \ce{1a}\ \ce{1b}\ \ce{1c}\ \ce{2e}\ \ce{3f}\ \ce{3g}\ \ce{4a}\ \ce{4b}\ \ce{4c}\ \ce{4d}\ \ce{4e}\ \ce{4f}\ \ce{4h}\ \ev2\ \ra1\ \ra2\ \ra3\ \ra4\ (est: 2022\==02\==07)
    \begin{longenum}
        \item Uso básico de objetos
        \begin{longenum}
            \item Instanciación
            \begin{longenum}
                \item \texttt{new}
                \item \texttt{getClass}
                \item \texttt{instanceof}
            \end{longenum}
            \item Referencias
            \begin{longenum}
                \item \texttt{null}
            \end{longenum}
            \item Comparación de objetos
            \begin{longenum}
                \item \texttt{equals}
                \item \texttt{hashCode}
            \end{longenum}
            \item Destrucción de objetos y recolección de basura
        \end{longenum}
        \item Clases y objetos básicos en Java
        \begin{longenum}
            \item Cadenas
            \begin{longenum}
                \item Inmutables
                \item Mutables
                \begin{longenum}
                    \item \texttt{StringBuffer}
                    \item \texttt{StringBuilder}
                    \item \texttt{StringTokenizer}
                \end{longenum}
                \item Conversión a \texttt{String}
                \item Concatenación de cadenas
                \item Comparación de cadenas
                \item Diferencias entre literales cadena y objetos \texttt{String}
            \end{longenum}
            \item Clases envolventes (\textit{wrapper})
            \begin{longenum}
                \item \textit{Boxing} y \textit{unboxing}
                \item \textit{Autoboxing} y \textit{autounboxing}
                \item La clase \texttt{Number}
            \end{longenum}
        \end{longenum}
        \item \textit{Arrays}
        \begin{longenum}
            \item Definición
            \item Declaración
            \item Creación
            \item Inicialización
            \item Acceso a elementos
            \item Longitud de un \_array\_
            \item Modificación de elementos
            \item \_Arrays\_ de tipos referencia
            \item Subtipado entre \_arrays\_
            \item \texttt{java.util.Arrays}
            \item Copia y redimensionado de \_arrays\_
            \begin{longenum}
                \item \texttt{clone}
                \item \texttt{System.arraycopy})
                \item \texttt{Arrays.copyOf}
            \end{longenum}
            \item Comparación de \textit{arrays}
            \begin{longenum}
                \item \texttt{Arrays.equals}
            \end{longenum}
            \item \_Arrays\_ multidimensionales
            \begin{longenum}
                \item Declaración
                \item Creación
                \item Inicialización
                \item \texttt{Arrays.deepEquals}
            \end{longenum}
        \end{longenum}
    \end{longenum}
    \item \textbf{\textsc{Diseño de clases en Java}} \ev2\ (est: 2022\==02\==14)
    \begin{longenum}
        \item Definición de clases
        \begin{longenum}
            \item Sintaxis básica
            \item Clases y paquetes
            \item Visibilidad de una clase
            \begin{longenum}
                \item Visibilidad predeterminada
                \item Visibilidad pública
            \end{longenum}
            \item Visibilidad de un miembro de una clase
        \end{longenum}
        \item Miembros de instancia
        \begin{longenum}
            \item Variables de instancia
            \begin{longenum}
                \item Acceso y modificación
                \item Variables de instancia finales
            \end{longenum}
            \item Métodos de instancia
            \begin{longenum}
                \item Invocación
                \item La sentencia \texttt{return}
                \item Referencia \texttt{this}
                \item Ámbito y resolución de identificadores
                \item Accesores y mutadores
                \item Sobrecarga
                \item Constructores
                \begin{longenum}
                    \item Sobrecarga de constructores
                    \item Constructor por defecto
                \end{longenum}
            \end{longenum}
        \end{longenum}
        \item Miembros estáticos
        \begin{longenum}
            \item Métodos estáticos
            \item Variables estáticas
        \end{longenum}
        \item Clases internas
        \begin{longenum}
            \item Clases internas anidadas
            \item Clases anidadas estáticas
        \end{longenum}
    \end{longenum}
    \item \textbf{\textsc{Relaciones entre clases en Java}} \ce{1a}\ \ce{1b}\ \ce{1c}\ \ce{3f}\ \ce{3g}\ \ce{4g}\ \ce{7a}\ \ce{7b}\ \ce{7c}\ \ce{7d}\ \ce{7e}\ \ce{7f}\ \ce{7g}\ \ce{7h}\ \ev2\ \ra1\ \ra3\ \ra4\ \ra7\ (est: 2022\==02\==21)
    \begin{longenum}
        \item Asociaciones básicas
        \begin{longenum}
            \item Agregación
            \item Composición
        \end{longenum}
        \item Generalización
        \begin{longenum}
            \item Declaración
            \item Subtipado entre tipos referencia
            \item Herencia
            \item La clase \texttt{Object}
            \item Visibilidad protegida
        \end{longenum}
        \item Polimorfismo
        \begin{longenum}
            \item El principio de sustitución de Liskov
            \begin{longenum}
                \item Ligadura temprana (\textit{early binding})
            \end{longenum}
            \item Sobreescritura de métodos
            \begin{longenum}
                \item Despacho dinámico (\textit{dynamic dispatch})
                \item Sobreescritura y visibilidad
                \item \texttt{super}
                \item Covarianza en el tipo de retorno
                \item Invarianza en el tipo de los argumentos
                \item Sobreescritura de \texttt{equals}
                \item Sobreescritura de \texttt{hashCode}
            \end{longenum}
        \end{longenum}
        \item Restricciones
        \begin{longenum}
            \item Clases y métodos abstractos
            \item Clases y métodos finales
        \end{longenum}
    \end{longenum}
    \item \textbf{\textsc{Programación modular (II)}} \ce{1a}\ \ce{1b}\ \ce{1c}\ \ce{3f}\ \ce{3g}\ \ce{4i}\ \ce{4j}\ \ev2\ \ra1\ \ra3\ \ra4\ (est: 2022\==02\==28)
    \begin{longenum}
        \item Interfaces
        \begin{longenum}
            \item Definición de interfaces
            \item Implementación de interfaces
            \item Las interfaces como tipos
            \item Herencia entre interfaces
            \item Métodos predeterminados
            \item Ejemplo: Interfaz \texttt{CharSequence}
            \item Ejemplo: Clonación de objetos
            \begin{longenum}
                \item \texttt{Cloneable}
                \item \texttt{Object.clone}
                \item Constructor de copia
            \end{longenum}
            \item Clases abstractas vs. interfaces
        \end{longenum}
        \item Paquetes y módulos
    \end{longenum}
    \item \textbf{\textsc{Programación genérica}} \ce{1a}\ \ce{1b}\ \ce{1c}\ \ce{3f}\ \ce{3g}\ \ce{6f}\ \ev2\ \ra1\ \ra3\ \ra6\ (est: 2022\==03\==07)
    \begin{longenum}
        \item Tipos genéricos
        \begin{longenum}
            \item Parámetros de tipo
            \item Argumentos de tipo
            \item Tipos crudos
        \end{longenum}
        \item Métodos genéricos
        \item Subtipos
        \begin{longenum}
            \item Parámetros de tipo acotados
            \item Clases genéricas, herencia y subtipos
            \begin{longenum}
                \item Covarianza
                \item Contravarianza
                \item Invarianza
            \end{longenum}
        \end{longenum}
        \item Inferencia de tipos
        \item Comodines
        \item Borrado de tipos
        \item Limitaciones
    \end{longenum}
    \item \textbf{\textsc{Control de excepciones en Java}} \ce{1a}\ \ce{1b}\ \ce{1c}\ \ce{3d}\ \ce{3f}\ \ce{3g}\ \ev2\ \ra1\ \ra3\ (est: 2022\==03\==14)
    \begin{longenum}
        \item Errores y excepciones
        \item El requisito «\textit{captura o especifica}»
        \begin{longenum}
            \item Tipos de excepciones
        \end{longenum}
        \item Captura y manejo de excepciones
        \begin{longenum}
            \item Bloque \texttt{try}
            \item Bloques \texttt{catch}
            \item Bloque \texttt{finally}
        \end{longenum}
        \item Excepciones y signaturas
        \item Lanzamiento de excepciones
        \begin{longenum}
            \item Excepciones encadenadas
            \item Creación de clases de excepción
        \end{longenum}
        \item Excepciones no chequeadas
        \item Ventajas de las excepciones
    \end{longenum}
    \item \textbf{\textsc{Java Collections Framework}} \ce{1a}\ \ce{1b}\ \ce{1c}\ \ce{3f}\ \ce{3g}\ \ce{6a}\ \ce{6b}\ \ce{6c}\ \ce{6d}\ \ce{6e}\ \ce{6f}\ \ev3\ \ra1\ \ra3\ \ra6\ (est: 2022\==03\==21)
    \begin{longenum}
        \item Colecciones y \textit{arrays}
        \item Arquitectura
        \item Tipos de colecciones
        \begin{longenum}
            \item Listas ordenadas
            \item Conjuntos
            \item Diccionarios
        \end{longenum}
        \item Listas
        \begin{longenum}
            \item \texttt{java.util.List}
            \begin{longenum}
                \item \texttt{java.util.ArrayList}
                \item \texttt{java.util.LinkedList}
                \item \texttt{java.util.Stack}
            \end{longenum}
        \end{longenum}
        \item Colas
        \begin{longenum}
            \item Interfaz \texttt{java.util.Queue}
            \begin{longenum}
                \item \texttt{java.util.ArrayDeque}
                \item \texttt{java.util.PriorityQueue}
                \item \texttt{java.util.LinkedList}
            \end{longenum}
            \item Interfaz \texttt{java.util.Deque}
            \begin{longenum}
                \item \texttt{java.util.ArrayDeque}
                \item \texttt{java.util.LinkedList}
            \end{longenum}
        \end{longenum}
        \item Conjuntos
        \begin{longenum}
            \item Interfaz \texttt{java.util.Set}
            \begin{longenum}
                \item \texttt{java.util.HashSet}
                \item \texttt{java.util.LinkedHashSet}
                \item \texttt{java.util.TreeSet}
            \end{longenum}
            \item Interfaz \texttt{java.util.SortedSet}
            \begin{longenum}
                \item \texttt{java.util.TreeSet}
            \end{longenum}
            \item Interfaz \texttt{java.util.NavigableSet}
            \begin{longenum}
                \item \texttt{java.util.TreeSet}
            \end{longenum}
        \end{longenum}
        \item Diccionarios
        \begin{longenum}
            \item Interfaz \texttt{java.util.Map}
            \begin{longenum}
                \item \texttt{java.util.HashMap}
                \item \texttt{java.util.LinkedHashMap}
                \item \texttt{java.util.TreeMap}
            \end{longenum}
            \item Interfaz \texttt{java.util.SortedMap}
            \begin{longenum}
                \item \texttt{java.util.TreeMap}
            \end{longenum}
            \item Interfaz \texttt{java.util.NavigableMap}
            \begin{longenum}
                \item \texttt{java.util.TreeMap}
            \end{longenum}
        \end{longenum}
    \end{longenum}
    \item \textbf{\textsc{Estructuras de datos lineales}} \ce{1a}\ \ce{1b}\ \ce{1c}\ \ce{3f}\ \ce{3g}\ \ce{6a}\ \ev3\ \ra1\ \ra3\ \ra6\ (est: 2022\==03\==28)
    \begin{longenum}
        \item Acceso secuencial
        \begin{longenum}
            \item Listas
            \begin{longenum}
                \item Enlazadas
                \item Doblemente enlazadas
            \end{longenum}
            \item Pilas
            \item Colas
        \end{longenum}
        \item Acceso directo
        \begin{longenum}
            \item Tablas \textit{hash}
        \end{longenum}
    \end{longenum}
    \item \textbf{\textsc{Ordenación y búsqueda}} \ce{1a}\ \ce{1b}\ \ce{1c}\ \ce{3f}\ \ce{3g}\ \ev3\ \ra1\ \ra3\ \ra6\ (est: 2022\==04\==04)
    \begin{longenum}
        \item Algoritmos de búsqueda
        \begin{longenum}
            \item Búsqueda secuencial
            \item Búsqueda dicotómica
        \end{longenum}
        \item Algoritmos de ordenación
        \begin{longenum}
            \item Inserción directa
            \item Selección directa
            \item Burbuja
            \item \textit{Quicksort}
            \item \textit{Mergesort}
        \end{longenum}
        \item Tablas \textit{Hash}
    \end{longenum}
    \item \textbf{\textsc{Estructuras de datos no lineales}} \ce{1a}\ \ce{1b}\ \ce{1c}\ \ce{3f}\ \ce{3g}\ \ev3\ \ra1\ \ra3\ \ra6\ (est: 2022\==04\==18)
    \begin{longenum}
        \item Árboles
        \begin{longenum}
            \item Binarios
            \begin{longenum}
                \item Recorridos
                \begin{longenum}
                    \item Preorden
                    \item Inorden
                    \item Postorden
                \end{longenum}
            \end{longenum}
            \item De búsqueda
            \item Montículos
            \begin{longenum}
                \item Algoritmo de ordenación
            \end{longenum}
            \item Generales
            \begin{longenum}
                \item Recorrido en profundidad
                \item Recorrido en anchura
            \end{longenum}
        \end{longenum}
        \item Grafos
        \begin{longenum}
            \item Algoritmo de Dijkstra
            \item Algoritmo de Floyd
        \end{longenum}
    \end{longenum}
    \item \textbf{\textsc{Calidad (II)}} \ce{1a}\ \ce{1b}\ \ce{1c}\ \ce{3f}\ \ce{3g}\ \ev3\ \ra1\ \ra3\ (est: 2022\==04\==25)
    \begin{longenum}
        \item Pruebas automáticas
        \begin{longenum}
            \item JUnit
        \end{longenum}
        \item Documentación
        \begin{longenum}
            \item Interna
            \begin{longenum}
                \item Reglas de estilo
                \item Google Java Format
            \end{longenum}
            \item Externa
            \begin{longenum}
                \item Javadoc
            \end{longenum}
        \end{longenum}
    \end{longenum}
    \item \textbf{\textsc{Gestión de bases de datos relacionales}} \ce{1a}\ \ce{1b}\ \ce{1c}\ \ce{3f}\ \ce{3g}\ \ce{8a}\ \ce{8b}\ \ce{8c}\ \ce{8d}\ \ce{8e}\ \ce{8f}\ \ce{8g}\ \ce{8h}\ \ce{9a}\ \ce{9b}\ \ce{9c}\ \ce{9d}\ \ce{9e}\ \ce{9f}\ \ce{9g}\ \ev3\ \ra1\ \ra3\ \ra6\ \ra8\ \ra9\ (est: 2022\==05\==02)
    \begin{longenum}
        \item Controlador JDBC
        \begin{longenum}
            \item Instalación
            \item \texttt{CLASSPATH}
            \item Carga
        \end{longenum}
        \item Establecimiento de conexiones
        \item Recuperación de información
        \begin{longenum}
            \item Ejecución de consultas
            \item Selección de registros
            \item Uso de parámetros
        \end{longenum}
        \item Manipulación de la información
        \begin{longenum}
            \item Altas, bajas y modificaciones
        \end{longenum}
    \end{longenum}
    \item \textbf{\textsc{Programación de interfaces gráficas de usuario}} \ce{1a}\ \ce{1b}\ \ce{1c}\ \ce{3f}\ \ce{3g}\ \ce{5f}\ \ce{5g}\ \ce{5h}\ \ev3\ \ra1\ \ra3\ \ra5\ \ra6\ (est: 2022\==05\==09)
    \begin{longenum}
        \item JFC y Swing
        \item Componentes de Swing
        \item Contenedores de nivel superior
        \begin{longenum}
            \item \texttt{JFrame}
            \item \texttt{JDialog}
            \item \texttt{JApplet}
        \end{longenum}
        \item \texttt{JComponent}
        \item Componentes de texto
        \begin{longenum}
            \item \texttt{JTextComponent}
        \end{longenum}
        \item Marcos
        \item Etiquetas
        \item Botones
        \item Arquitectura de modelos de Swing
    \end{longenum}
    \item \textbf{\textsc{Entrada y salida de información}} \ce{1a}\ \ce{1b}\ \ce{1c}\ \ce{3f}\ \ce{3g}\ \ce{5a}\ \ce{5b}\ \ce{5c}\ \ce{5d}\ \ce{5e}\ \ce{6h}\ \ce{6i}\ \opcional\ \ra1\ \ra3\ \ra5\ \ra6\
    \begin{longenum}
        \item La consola
        \begin{longenum}
            \item Entrada desde teclado
            \item Salida a pantalla
        \end{longenum}
        \item Archivos de datos
        \begin{longenum}
            \item Registros
            \item Apertura y cierre de archivos
            \item Modos de acceso
            \item Lectura y escritura de información en archivos
        \end{longenum}
        \item Manipulación de archivos XML
        \item Serialización de objetos
        \item Sistemas de archivos
        \begin{longenum}
            \item Manipulación de los sistemas de archivos
            \item Creación y eliminación de archivos y directorios
        \end{longenum}
    \end{longenum}
    \item \textbf{\textsc{Complejidad algorítmica}} \opcional\
    \begin{longenum}
        \item Introducción
        \item Principio de invarianza
        \item La notación asintótica \textit{O(f(n))}
        \item Órdenes de complejidad
        \item Operaciones entre órdenes de complejidad
        \begin{longenum}
            \item Regla de la suma
            \item Regla del producto
        \end{longenum}
        \item Reglas prácticas para el cálculo de la eficiencia
        \item Resolución de recurrencias
        \begin{longenum}
            \item Reducción de problemas mediante sustracción
            \item Reducción de problemas mediante división
        \end{longenum}
    \end{longenum}
    \item \textbf{\textsc{Metodología de la programación}} \ce{1a}\ \ce{1b}\ \ce{1c}\ \ce{3f}\ \ce{3g}\ \opcional\ \ra1\ \ra3\ \ra6\
    \begin{longenum}
        \item Ciclo de vida
        \item Especificación e implementación
        \item Verificación y validación de programas
        \begin{longenum}
            \item Demostraciones por inducción
        \end{longenum}
        \item Programación funcional
        \begin{longenum}
            \item Especificaciones formales
            \begin{longenum}
                \item Como cálculo
            \end{longenum}
            \item Derivación de programas
            \begin{longenum}
                \item Diseño recursivo
                \begin{longenum}
                    \item Recursividad final
                    \item Técnicas de inmersión \opcional\
                \end{longenum}
            \end{longenum}
        \end{longenum}
        \item Programación imperativa
        \begin{longenum}
            \item Especificaciones formales
            \begin{longenum}
                \item Como modificación de estados
            \end{longenum}
            \item Derivación de programas
            \begin{longenum}
                \item Diseño iterativo
                \begin{longenum}
                    \item Invariante de un bucle
                    \item Transformación de recursividad final a iterativo
                \end{longenum}
            \end{longenum}
        \end{longenum}
        \item El lenguaje Dafny \opcional\
    \end{longenum}
    \item \textbf{\textsc{Principios y patrones de diseño}} \ce{1a}\ \ce{1b}\ \ce{1c}\ \ce{3f}\ \ce{3g}\ \opcional\ \ra1\ \ra3\
    \begin{longenum}
        \item Principios de diseño
        \begin{longenum}
            \item Encapsulación y ocultación de información
            \item Diseño orientado a interfaces
            \item Principios \textit{SOLID}
            \begin{longenum}
                \item SRP: Principio de responsabilidad única
                \item OCP: Principio de abierto/cerrado
                \item LSP: Principio de sustitución de Liskov
                \item ISP: Principio de segregación de la interfaz
                \item DIP: Principio de inversión de dependencias
            \end{longenum}
            \item Principio del Menor Conocimiento (o Ley de Demeter)
        \end{longenum}
        \item Patrones de diseño
        \begin{longenum}
            \item De creación
            \item Estructurales
            \item De comportamiento
        \end{longenum}
    \end{longenum}
\end{longenum}
