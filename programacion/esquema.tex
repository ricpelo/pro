\begin{longenum}
    \item \textbf{\textsc{Introducción}} \ev1\ (est: 2025\==09\==15)
    \begin{longenum}
        \item Conceptos básicos
        \begin{longenum}
            \item Informática
            \begin{longenum}
                \item Procesamiento automático
            \end{longenum}
            \item Ordenador
            \begin{longenum}
                \item Definición
                \item Funcionamiento básico
                \begin{longenum}
                    \item Elementos funcionales
                    \item Ciclo de instrucción
                    \item Representación de información
                    \begin{longenum}
                        \item Codificación interna
                        \begin{longenum}
                            \item Sistema binario
                        \end{longenum}
                        \item Codificación externa
                        \begin{longenum}
                            \item ASCII
                            \item Unicode
                        \end{longenum}
                    \end{longenum}
                \end{longenum}
            \end{longenum}
            \item Problema
            \begin{longenum}
                \item Generalización
                \item Ejemplares de un problema
                \item Dominio de definición
                \item Jerarquías de generalización
            \end{longenum}
            \item Algoritmo
            \begin{longenum}
                \item Definición
                \item Características
                \item Representación
                \begin{longenum}
                    \item Ordinograma
                    \item Pseudocódigo
                \end{longenum}
                \item Cualidades deseables
                \item Computabilidad
                \item Corrección
                \item Complejidad
            \end{longenum}
            \item Programa
            \item Lenguaje de programación
        \end{longenum}
        \item Paradigmas de programación
        \begin{longenum}
            \item Definición
            \item Imperativo
            \begin{longenum}
                \item Estructurado
                \item Orientado a objetos
            \end{longenum}
            \item Declarativo
            \begin{longenum}
                \item Funcional
                \item Lógico
                \item De bases de datos
            \end{longenum}
        \end{longenum}
        \item Resolución de problemas mediante programación
        \begin{longenum}
            \item Especificación
            \item Análisis del problema
            \item Diseño del algoritmo
            \item Verificación
            \item Estudio de la eficiencia
            \item Codificación
            \begin{longenum}
                \item Implementación
            \end{longenum}
            \item Traducción y ejecución
            \item Pruebas
            \item Depuración
            \item Documentación
            \item Mantenimiento
            \item Ingeniería del software
        \end{longenum}
    \end{longenum}
    \item \textbf{\textsc{Lenguajes de programación}} \ev1\ (est: 2025\==09\==22)
    \begin{longenum}
        \item Definición
        \begin{longenum}
            \item Sintaxis
            \begin{longenum}
                \item Notación EBNF
            \end{longenum}
            \item Semántica estática
            \item Semántica dinámica
        \end{longenum}
        \item Evolución histórica \opcional\
        \item Clasificación
        \begin{longenum}
            \item Por nivel
            \item Por generación
            \item Por propósito
            \item Por paradigma
        \end{longenum}
        \item Traductores e intérpretes
        \begin{longenum}
            \item Traductores
            \item Compiladores
            \begin{longenum}
                \item Ensambladores
            \end{longenum}
            \item Intérpretes
            \begin{longenum}
                \item Interactivos (\textit{REPL})
            \end{longenum}
        \end{longenum}
        \item Entornos integrados de desarrollo
        \begin{longenum}
            \item Definición
            \item Editores de textos
            \item Editores vs. IDE
            \item Thonny
        \end{longenum}
    \end{longenum}
    \item \textbf{\textsc{Expresiones}} \ev1\ (est: 2025\==09\==29)
    \begin{longenum}
        \item El lenguaje de programación Python
        \begin{longenum}
            \item Historia
            \item Características principales
            \item Instalación
            \item Funcionamiento del intérprete
            \begin{longenum}
                \item Entrar y salir del intérprete interactivo
            \end{longenum}
        \end{longenum}
        \item Elementos de un programa
        \begin{longenum}
            \item Expresiones y sentencias
            \item Sintaxis y semántica de las expresiones
        \end{longenum}
        \item Valores
        \begin{longenum}
            \item Información, datos, tipos y valores
            \item Evaluación de expresiones
            \item Expresión canónica y forma normal
            \item Formas normales y evaluación
            \item Árboles sintácticos y evaluación
            \item Literales
            \item Identificadores
        \end{longenum}
        \item Operaciones
        \begin{longenum}
            \item Clasificación
            \item Operadores
            \begin{longenum}
                \item Aridad de operadores
                \item Notación de los operadores
                \item Paréntesis
                \item Prioridad de operadores
                \item Asociatividad de operadores
                \item Paréntesis y orden de evaluación
                \item Tipos de operandos
            \end{longenum}
            \item Funciones
            \begin{longenum}
                \item Funciones con varios parámetros
                \item Evaluación de expresiones con funciones
                \item Composición de operaciones y funciones
            \end{longenum}
            \item Métodos
        \end{longenum}
    \end{longenum}
    \item \textbf{\textsc{Programación funcional (I)}} \ev1\ (est: 2025\==10\==06)
    \begin{longenum}
        \item Introducción
        \begin{longenum}
            \item Concepto
            \item Transparencia referencial
            \begin{longenum}
                \item Efectos laterales
            \end{longenum}
            \item Modelo de ejecución
            \begin{longenum}
                \item Modelo de sustitución
            \end{longenum}
        \end{longenum}
        \item Tipos de datos
        \begin{longenum}
            \item Concepto
            \item Tipo de un dato
            \item \texttt{type}
            \item Sistemas de tipos
            \begin{longenum}
                \item Errores de tipos
                \item Tipado fuerte vs. débil
            \end{longenum}
            \item Tipos de datos básicos
            \begin{longenum}
                \item Números
                \item Cadenas
                \item Tuplas
                \item Funciones
            \end{longenum}
            \item Conversión de tipos
        \end{longenum}
        \item Operaciones predefinidas
        \begin{longenum}
            \item Operadores predefinidos
            \begin{longenum}
                \item Operadores aritméticos
                \item Operadores de cadenas y tuplas
            \end{longenum}
            \item Funciones predefinidas
            \begin{longenum}
                \item Funciones matemáticas y módulos
                \begin{longenum}
                    \item El módulo \texttt{operator}
                \end{longenum}
            \end{longenum}
            \item Métodos predefinidos
        \end{longenum}
        \item Álgebra de Boole
        \begin{longenum}
            \item El tipo de dato \textit{booleano}
            \item Operadores relacionales
            \item Operadores lógicos
            \begin{longenum}
                \item Tablas de verdad
            \end{longenum}
            \item Axiomas
            \item Teoremas fundamentales
            \item Equivalencia lógica
            \item El operador ternario
        \end{longenum}
        \item Otros conceptos sobre operaciones
        \begin{longenum}
            \item Tipos polimórficos y operaciones polimórficas
            \item Sobrecarga de operaciones
            \item Equivalencia entre formas de operaciones
            \item Igualdad de operaciones
        \end{longenum}
    \end{longenum}
    \item \textbf{\textsc{Programación funcional (II)}} \ev1\ (est: 2025\==10\==13)
    \begin{longenum}
        \item Definiciones
        \begin{longenum}
            \item Identificadores y ligaduras (\textit{binding})
            \begin{longenum}
                \item Ligaduras irrompibles
                \item Inmutabilidad
                \item Reglas léxicas
                \item Tipo de un identificador
                \item Anotaciones de tipo en definiciones
            \end{longenum}
            \item Espacios de nombres
            \item Marcos (\textit{frames})
            \item Evaluación de expresiones con identificadores
            \begin{longenum}
                \item Resolución de identificadores
            \end{longenum}
        \end{longenum}
        \item \textit{Scripts}
        \item Documentación interna
        \begin{longenum}
            \item Concepto
            \item Comentarios
            \item Identificadores significativos
            \item Estándares de codificación
            \begin{longenum}
                \item PEP 8
                \item \texttt{pylint}
            \end{longenum}
        \end{longenum}
    \end{longenum}
    \item \textbf{\textsc{Abstracciones funcionales}} \ev1\ (est: 2025\==10\==20)
    \begin{longenum}
        \item Abstracciones lambda
        \begin{longenum}
            \item Expresiones lambda
            \item Parámetros y cuerpos
            \item Aplicación funcional
            \begin{longenum}
                \item Evaluación de una aplicación funcional
                \item Funciones con nombre
            \end{longenum}
            \item Identificadores locales y libres de una expresión lambda
            \item Anotaciones de tipo en expresiones lambda
        \end{longenum}
        \item Ámbitos
        \begin{longenum}
            \item Ámbitos léxicos
            \item Ámbito de una definición y de una ligadura
            \begin{longenum}
                \item Visibilidad
                \item Tiempo de vida
                \item Almacenamiento
            \end{longenum}
            \item Ámbito de un identificador
            \item Ámbito de un parámetro
            \item Ámbito de un identificador libre
        \end{longenum}
        \item Funciones recursivas \opcional\
        \begin{longenum}
            \item Definición
            \item Casos base y casos recursivos
            \item El factorial
            \item Diseño de funciones recursivas
            \begin{longenum}
                \item Identificación de casos base
                \item Descomposición (reducción) del problema
                \item Pensamiento optimista
            \end{longenum}
            \item Recursividad lineal
            \begin{longenum}
                \item Procesos recursivos lineales \opcional\
                \item Procesos iterativos lineales \opcional\
            \end{longenum}
            \item Recursividad múltiple
            \item Recursividad final y no final \opcional\
        \end{longenum}
        \item Abstracciones funcionales
        \begin{longenum}
            \item Pureza
            \item Las funciones como abstracciones
            \begin{longenum}
                \item Especificaciones de funciones
            \end{longenum}
        \end{longenum}
    \end{longenum}
    \item \textbf{\textsc{Evaluación}} \ev1\ (est: 2025\==10\==27)
    \begin{longenum}
        \item El modelo de entorno (\textit{environment})
        \begin{longenum}
            \item Ámbitos, marcos y entornos
            \item Evaluación de expresiones con entornos
            \item Evaluación de expresiones lambda con entornos
            \begin{longenum}
                \item Ligaduras \textit{sombreadas}
                \item Renombrado de parámetros
                \item Visualización en \textit{Pythontutor}
            \end{longenum}
        \end{longenum}
        \item Resolución de atributos de objetos
        \item La pila de control
        \item Estrategias de evaluación \opcional\
        \begin{longenum}
            \item Orden de evaluación
            \begin{longenum}
                \item Orden aplicativo
                \item Orden normal
            \end{longenum}
            \item Composición de funciones
            \item Evaluación estricta y no estricta
        \end{longenum}
    \end{longenum}
    \item \textbf{\textsc{Programación imperativa (I)}} \ev1\ (est: 2025\==11\==03)
    \begin{longenum}
        \item Modelo de ejecución
        \begin{longenum}
            \item Máquina de estados
            \item Sentencias
            \item Secuencia de sentencias
        \end{longenum}
        \item Asignación destructiva
        \begin{longenum}
            \item Valores y referencias
            \item Variables
            \item Estado
            \item Marcos en programación imperativa
            \item Sentencia de asignación
            \item La sentencia \texttt{del}
            \item Alias de variables y valores idénticos
            \item Recolección de basura
            \item Evaluación de expresiones con variables
            \item Tipado estático vs. dinámico
            \item Asignación compuesta
            \item Asignación múltiple
            \item Constantes
        \end{longenum}
        \item Saltos
        \begin{longenum}
            \item Incondicionales
            \item Condicionales
        \end{longenum}
    \end{longenum}
    \item \textbf{\textsc{Programación imperativa (II)}} \ev1\ (est: 2025\==11\==10)
    \begin{longenum}
        \item Mutabilidad
        \begin{longenum}
            \item Estado de un dato
            \item Tipos mutables e inmutables
            \begin{longenum}
                \item Valores inmutables
                \begin{longenum}
                    \item Secuencias
                \end{longenum}
                \item Valores mutables: listas
            \end{longenum}
            \item Alias de variables y valores mutables
            \item Identidad
            \begin{longenum}
                \item \texttt{is}
            \end{longenum}
        \end{longenum}
        \item Cambios de estado ocultos
        \begin{longenum}
            \item Funciones puras
            \item Funciones impuras
            \item Efectos laterales
            \item Ejecución de \textit{scripts} por lotes
            \begin{longenum}
                \item Argumentos de la línea de órdenes
            \end{longenum}
        \end{longenum}
        \item Entrada y salida por consola
        \begin{longenum}
            \item \texttt{print}
            \begin{longenum}
                \item Paso de argumentos por palabras clave
                \item El valor \texttt{None}
            \end{longenum}
            \item \texttt{input}
        \end{longenum}
    \end{longenum}
    \item \textbf{\textsc{Programación estructurada}} \ev1\ (est: 2025\==11\==17)
    \begin{longenum}
        \item Aspectos teóricos de la programación estructurada
        \begin{longenum}
            \item Programación estructurada
            \item Programa restringido
            \item Programa propio
            \item Estructura
            \item Programa estructurado
            \begin{longenum}
                \item Ventajas de los programas estructurados
            \end{longenum}
            \item Teorema de Böhm-Jacopini
        \end{longenum}
        \item Estructuras básicas de control en Python
        \begin{longenum}
            \item Secuencia
            \item Selección
            \item Iteración
            \item Otras sentencias de control
            \begin{longenum}
                \item \texttt{break}
                \item \texttt{continue}
                \item Excepciones
                \begin{longenum}
                    \item Gestión de excepciones
                \end{longenum}
            \end{longenum}
        \end{longenum}
        \item Metodología de la programación estructurada
        \begin{longenum}
            \item Recursos abstractos
            \item Diseño descendente
            \item Refinamiento sucesivo
        \end{longenum}
    \end{longenum}
    \item \textbf{\textsc{Programación procedimental}} \ev1\ (est: 2025\==11\==24)
    \begin{longenum}
        \item Conceptos básicos
        \begin{longenum}
            \item Procedimientos
            \item El paradigma de programación procedimental
            \item Procedimientos y refinamiento sucesivo
        \end{longenum}
        \item Funciones imperativas
        \begin{longenum}
            \item Introducción
            \item Definición de funciones imperativas
            \item Llamadas a funciones imperativas
            \item Paso de argumentos
            \item La sentencia \texttt{return}
            \item Ámbito de variables
            \begin{longenum}
                \item Variables locales
                \item Variables globales
                \begin{longenum}
                    \item \texttt{global}
                    \item Efectos laterales
                \end{longenum}
            \end{longenum}
            \item \textit{Docstrings}
            \begin{longenum}
                \item \texttt{pydoc}
            \end{longenum}
        \end{longenum}
        \item Funciones locales a funciones
        \begin{longenum}
            \item Definición
            \item \texttt{nonlocal}
        \end{longenum}
    \end{longenum}
    \item \textbf{\textsc{Procesamiento de datos de alto nivel}} \ev2\ (est: 2026\==01\==07)
    \begin{longenum}
        \item Colecciones
        \begin{longenum}
            \item Composición
            \item Conceptos básicos
            \item Clasificación
            \item \textit{Hashables}
        \end{longenum}
        \item Iterables e iteradores
        \begin{longenum}
            \item Iterables
            \item Iteradores
            \begin{longenum}
                \item El bucle \texttt{for}
                \item El módulo \texttt{itertools}
                \item \texttt{zip}
                \item \texttt{reversed}
            \end{longenum}
        \end{longenum}
        \item Funciones de orden superior
        \begin{longenum}
            \item Concepto
            \item \texttt{map}
            \item \texttt{filter}
            \item \texttt{reduce}
            \item Expresiones generadoras
        \end{longenum}
    \end{longenum}
    \item \textbf{\textsc{Secuencias}} \ev2\ (est: 2026\==01\==14)
    \begin{longenum}
        \item Concepto de secuencia
        \begin{longenum}
            \item Definición
            \item Operaciones comunes
        \end{longenum}
        \item Inmutables
        \begin{longenum}
            \item Cadenas (\texttt{str})
            \begin{longenum}
                \item Formateado de cadenas
                \item Expresiones regulares
            \end{longenum}
            \item Tuplas
            \item Rangos
        \end{longenum}
        \item Mutables
        \begin{longenum}
            \item Listas
            \begin{longenum}
                \item Listas por compresión
            \end{longenum}
            \item Operaciones mutadoras
        \end{longenum}
    \end{longenum}
    \item \textbf{\textsc{Colecciones no secuenciales}} \ev2\ (est: 2026\==01\==21)
    \begin{longenum}
        \item Conjuntos (\texttt{set} y \texttt{frozenset})
        \begin{longenum}
            \item Definición
            \item Conjuntos por comprensión
            \item Operaciones
            \item Operaciones sobre conjuntos mutables
            \item Recorrido de conjuntos
        \end{longenum}
        \item Diccionarios (\texttt{dict})
        \begin{longenum}
            \item Definición
            \item Diccionarios por comprensión
            \item Operaciones
            \item Recorrido de diccionarios
        \end{longenum}
        \item Documentos XML
        \begin{longenum}
            \item Definición
            \item Acceso
            \item Modificación
        \end{longenum}
    \end{longenum}
    \item \textbf{\textsc{Entrada y salida por archivos}} \ev2\ (est: 2026\==01\==28)
    \begin{longenum}
        \item Introducción
        \item Operaciones de apertura y cierre
        \begin{longenum}
            \item \texttt{open}
            \item \texttt{close}
            \item Gestores de contexto
        \end{longenum}
        \item Operaciones de lectura
        \begin{longenum}
            \item \texttt{read}
            \item \texttt{readline}
            \item \texttt{readlines}
        \end{longenum}
        \item Operaciones de escritura
        \begin{longenum}
            \item \texttt{write}
            \item \texttt{writelines}
        \end{longenum}
        \item Otras operaciones
        \begin{longenum}
            \item \texttt{seek} y \texttt{tell}
        \end{longenum}
    \end{longenum}
    \item \textbf{\textsc{Programación modular}} \ev2\ (est: 2026\==02\==04)
    \begin{longenum}
        \item Introducción
        \begin{longenum}
            \item Modularidad
            \item Descomposición de problemas
            \item Beneficios de la modularidad
        \end{longenum}
        \item Diseño modular
        \begin{longenum}
            \item Creadores y usuarios
            \item Partes de un módulo
            \begin{longenum}
                \item Interfaz
                \begin{longenum}
                    \item Especificación
                \end{longenum}
                \item Implementación
            \end{longenum}
            \item Diagramas de estructura
        \end{longenum}
        \item Programación modular en Python
        \begin{longenum}
            \item \textit{Scripts} como módulos
            \item Importación de módulos
            \item Módulos como \textit{scripts}
            \item La librería estándar
            \item Paquetes \opcional\
            \item Documentación interna \opcional\
        \end{longenum}
        \item Criterios de descomposición modular
        \begin{longenum}
            \item Abstracción
            \item Ocultación de información
            \item Independencia funcional
            \begin{longenum}
                \item Cohesión
                \item Acoplamiento
            \end{longenum}
            \item Reusabilidad
        \end{longenum}
    \end{longenum}
    \item \textbf{\textsc{Abstracción de datos}} \ev2\ (est: 2026\==02\==11)
    \begin{longenum}
        \item Introducción
        \begin{longenum}
            \item Introducción
            \item Tipos abstractos de datos
        \end{longenum}
        \item Especificaciones
        \begin{longenum}
            \item Sintaxis
            \item Operaciones
            \item Ejemplos
        \end{longenum}
        \item Implementaciones
        \begin{longenum}
            \item Implementaciones
        \end{longenum}
        \item Niveles y barreras de abstracción
        \begin{longenum}
            \item Niveles de abstracción
            \item Barreras de abstracción
            \item Propiedades de los datos
        \end{longenum}
        \item Las funciones como datos
        \begin{longenum}
            \item Clausuras
            \item Representación funcional
            \item Estado interno
            \item Paso de mensajes
            \item Especificación de datos abstractos con estado interno
        \end{longenum}
        \item Abstracción de datos y modularidad
        \begin{longenum}
            \item El tipo abstracto como módulo
        \end{longenum}
    \end{longenum}
    \item \textbf{\textsc{Programación orientada a objetos}} \ev2\ (est: 2026\==02\==18)
    \begin{longenum}
        \item Introducción
        \begin{longenum}
            \item Recapitulación
            \item Objetos
        \end{longenum}
        \item Conceptos básicos
        \begin{longenum}
            \item Atributos
            \item Clases
            \begin{longenum}
                \item Instancias
            \end{longenum}
            \item Estado
            \begin{longenum}
                \item Variables de instancia
            \end{longenum}
            \item La antisimetría dato-objeto
        \end{longenum}
        \item Paso de mensajes
        \begin{longenum}
            \item Resolución de atributos
            \item Ejecución de métodos
            \item Definición de métodos
            \begin{longenum}
                \item Entorno durante la ejecución de métodos
            \end{longenum}
            \item Métodos \textit{mágicos} y constructores
        \end{longenum}
        \item Identidad e igualdad
        \begin{longenum}
            \item Identidad
            \item Igualdad
            \begin{longenum}
                \item \texttt{\_\_eq\_\_}
                \item \texttt{\_\_hash\_\_}
            \end{longenum}
            \item Otros métodos mágicos
            \begin{longenum}
                \item \texttt{\_\_repr\_\_}
                \item \texttt{\_\_str\_\_}
            \end{longenum}
        \end{longenum}
        \item Encapsulación
        \begin{longenum}
            \item La encapsulación como mecanismo de agrupamiento
            \item La encapsulación como mecanismo de protección de datos
            \begin{longenum}
                \item Visibilidad
                \item Accesores y mutadores
                \item Invariantes de clase
                \item Interfaz y especificación de una clase
                \item Asertos
            \end{longenum}
        \end{longenum}
        \item Miembros de clase
        \begin{longenum}
            \item Variables de clase
            \item Métodos estáticos
        \end{longenum}
    \end{longenum}
    \item \textbf{\textsc{Relaciones entre clases}} \ev2\ (est: 2026\==02\==25)
    \begin{longenum}
        \item Relaciones básicas
        \begin{longenum}
            \item Introducción
            \item Asociación
            \item Dependencia
            \item Agregación
            \item Composición
        \end{longenum}
        \item Herencia
        \begin{longenum}
            \item Generalización
            \item Modos
            \begin{longenum}
                \item Herencia simple
                \item Visibilidad de miembros y herencia
                \begin{longenum}
                    \item Visibilidad protegida
                \end{longenum}
                \item La clase \texttt{object}
                \item Herencia múltiple
            \end{longenum}
        \end{longenum}
        \item Polimorfismo
        \begin{longenum}
            \item Concepto
            \item Principio de sustitución de Liskov
            \item \textit{Duck typing}
            \item Sobreescritura de métodos
            \begin{longenum}
                \item Despacho dinámico (\textit{dynamic dispatch})
            \end{longenum}
            \item Ligadura dinámica
            \item \texttt{super}
            \item Sobreescritura de constructores
            \item Igualdad polimórfica
            \item Sobreescritura de \texttt{\_\_eq\_\_}
        \end{longenum}
        \item Abstracción
        \begin{longenum}
            \item Clases abstractas y métodos abstractos
        \end{longenum}
        \item Herencia vs. composición
    \end{longenum}
    \item \textbf{\textsc{Programación de interfaces gráficas de usuario}} \ev2\ (est: 2026\==03\==04)
    \begin{longenum}
        \item Introducción a Tkinter
        \begin{longenum}
            \item ¿Qué es Tkinter?
            \item Instalación y primeras pruebas
            \item La ventana principal (Tk)
            \item El bucle principal (mainloop())
        \end{longenum}
        \item Widgets básicos
        \begin{longenum}
            \item Label, Button, Entry, Text, Checkbutton, Radiobutton
            \item Atributos comunes: texto, color, fuente, tamaño
            \item Métodos útiles: get(), insert(), delete()
        \end{longenum}
        \item Layout y organización de la interfaz
        \begin{longenum}
            \item Geometría con pack(), grid() y place()
            \item Uso de Frame para dividir la ventana
            \item Diseño responsive básico
        \end{longenum}
        \item Eventos y funciones asociadas
        \begin{longenum}
            \item Asociar funciones a eventos (callbacks)
            \item Uso de command=
            \item Eventos con bind()
            \item Variables de control (StringVar, IntVar, etc.)
        \end{longenum}
    \end{longenum}
    \item \textbf{\textsc{Bases de datos orientadas a objetos}} \ev2\ (est: 2026\==03\==11)
    \begin{longenum}
        \item Introducción a las bases de datos orientadas a objetos (OODB)
        \begin{longenum}
            \item Conceptos clave
            \begin{longenum}
                \item Objetos, clases y herencia
                \item Persistencia de objetos
                \item Integración de datos y comportamiento
            \end{longenum}
            \item Comparación con RDBMS
            \begin{longenum}
                \item Modelos de datos relacionales vs. orientados a objetos
                \item Ventajas y desventajas de OODB
                \item Escenarios donde es preferible usar una OODB
            \end{longenum}
            \item Casos de uso comunes
            \begin{longenum}
                \item Aplicaciones de simulación
                \item Modelos de datos complejos
                \item Aplicaciones con estructuras de datos jerárquicas
            \end{longenum}
        \end{longenum}
        \item Persistencia de objetos en Python
        \begin{longenum}
            \item Serialización y deserialización de objetos (módulo \texttt{pickle})
        \end{longenum}
        \item Conceptos de bases de datos orientadas a objetos
        \begin{longenum}
            \item Modelado de una OODB
            \begin{longenum}
                \item Clases como tablas
                \item Relaciones entre objetos (1 a 1, 1 a muchos, muchos a muchos)
                \item Polimorfismo en la base de datos
            \end{longenum}
            \item Implementación básica en Python
            \begin{longenum}
                \item Guardar objetos en una estructura persistente
                \item Simular una base de datos orientada a objetos sencilla
            \end{longenum}
            \item Limitaciones de una implementación casera
            \begin{longenum}
                \item Consistencia de datos
                \item Concurrencia
                \item Integridad referencial
            \end{longenum}
        \end{longenum}
        \item Introducción a Zope Object Database (ZODB)
        \begin{longenum}
            \item ¿Qué es ZODB?
            \begin{longenum}
                \item Ventajas de usar ZODB
                \item Comparación con otras soluciones de bases de datos
            \end{longenum}
            \item Instalación y configuración
            \begin{longenum}
                \item Estructura básica de un proyecto con ZODB
            \end{longenum}
            \item Persistencia de objetos en ZODB
            \begin{longenum}
                \item Crear una base de datos en ZODB
                \item Guardar, actualizar y eliminar objetos
                \item Administración de la base de datos y control de versiones
            \end{longenum}
        \end{longenum}
    \end{longenum}
    \item \textbf{\textsc{Bases de datos relacionales}} \ce{9a}\ \ce{9b}\ \ce{9c}\ \ce{9d}\ \ce{9e}\ \ce{9f}\ \ce{9g}\ \ev3\ \ra9\ (est: 2025\==04\==23)
    \begin{longenum}
        \item Controlador JDBC
        \begin{longenum}
            \item Instalación
            \item \texttt{CLASSPATH}
            \begin{longenum}
                \item En consola
                \item En Visual Studio Code
            \end{longenum}
        \end{longenum}
        \item Programa de ejemplo
        \item Ejecutar programa
        \begin{longenum}
            \item En consola
            \item En Visual Studio Code
        \end{longenum}
        \item Establecimiento de conexiones
        \item Recuperación de información
        \begin{longenum}
            \item Ejecución de consultas
            \item Selección de registros
            \item Uso de parámetros
        \end{longenum}
        \item Manipulación de la información
        \begin{longenum}
            \item Altas, bajas y modificaciones
        \end{longenum}
    \end{longenum}
\end{longenum}
